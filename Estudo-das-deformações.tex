\section{Estudos das deformações}
	
	\subsection{Considerações iniciais}
	
	% \boldsymbol{g} para deixar o símbolo em negrito
	
	\begin{itemize}
		\item Configuração do sólido: posição ocupada pelos pontos em um determinado instante $t$;
		\item Descrever a configuração deformada ($V$) a partir de uma configuração de referência ($V^r$);
		\item Considerando um conjunto de pontos da Geometria Euclidiana ($E$), e seja, $(\utilde{\mathbf{e_1}},\;\utilde{\mathbf{e_2}},\;\utilde{\mathbf{e_3}})$ uma base do espaço vetorial da geometria clássica.
	\end{itemize}
	
	Podemos definir $(0,\;\utilde{\mathbf{e_1}},\;\utilde{\mathbf{e_2}},\;\utilde{\mathbf{e_3}})$ como um sistema de referência:
	
	// Inserir imagem
	
	Os vetores dos pontos de referência ($\utilde{\mathbf{x^r}}$) e na configuração deformada ($\utilde{\mathbf{x}}$) em relação aos versores da base podem ser expressos na notação de Einstein:
	
	\[\utilde{\mathbf{x^r}}=\utilde{\mathbf{x^r}}-\utilde{\mathbf{0}}=\sum_{i=1}^3x^{ri}\utilde{\mathbf{e_i}}=x^{r1}\utilde{\mathbf{e_1}}+x^{r2}\utilde{\mathbf{e_2}}+x^{r3}\utilde{\mathbf{e_3}}\]
	\[\utilde{\mathbf{x}}=\utilde{\mathbf{x}}-\utilde{\mathbf{0}}=\sum_{i=1}^3x^i\utilde{\mathbf{e_i}}=x^1\utilde{\mathbf{e_1}}+x^2\utilde{\mathbf{e_2}}+x^3\utilde{\mathbf{e_3}}\]
	
	Seja $\psi$ uma função que associa a posição de cada ponto na configuração $V^r$ a sua posição na configuração $V$. Tal aplicação é uma transformação de $V^r$ em $V$.
	
	\[x^1=\hat{x}^1(x^{r1},\;x^{r2},\;x^{r3})\]
	\[x^2=\hat{x}^2(x^{r1},\;x^{r2},\;x^{r3})\]
	\[x^3=\hat{x}^3(x^{r1},\;x^{r2},\;x^{r3})\]
	
	Aqui, o circunflexo denota \textit{em função de}, \textit{i.e.}, a coordenada $i$ da posição deformada está em função das coordenadas da posição de referência.
	
	Exemplo: Considere um sólido cuja seção no plano $\utilde{\mathbf{e_1}}$, $\utilde{\mathbf{e_2}}$ é dado por:
	
	//Inserir imagem
	
	Caracterize os seguintes campos de deslocamento:
	
	\begin{enumerate}[a)]
		\item Translação de corpo rígido de intensidade $\Delta$ na direção de $\utilde{\mathbf{e_1}}$;
		\item Rotação de corpo rígido em torno de $\utilde{\mathbf{e_3}}$ de intensidade $\varphi$.
	\end{enumerate}
	
	Resolução:
	
	\begin{enumerate}[a)]
		\item
			\[
			\utilde{\mathbf{u}}=
				\begin{Bmatrix}
				u_1 \\ u_2 \\ u_3
				\end{Bmatrix}
				=
				\begin{Bmatrix}
					\Delta \\ 0 \\ 0
				\end{Bmatrix}
			\]
			\[\utilde{\mathbf{u}}=\utilde{\mathbf{x}}-\utilde{\mathbf{x^r}}
			\implies
			\utilde{\mathbf{x}}=
			\begin{cases} x^1=x^{r1}+u_1=x^{r1}+\Delta \\ x^2=x^{r2}+u_2=x^{r2} \\ x^3=x^{r3}+u_3=x^{r3}
			\end{cases}
			\]
		\item
			//Inserir imagem
			
			A configuração de referência:
			\[x^{r1}=r\cdot \cos\theta\]
			\[x^{r2}=r\cdot\sin\theta\]
			\[x^{r3}=0\;\text{(ou }x^{r3}\text{ para deixar genérico})\]
			
			A configuração deformada (a partir da imagem):
			
			\[u^1=r\cdot\cos(\varphi+\theta)-r\cdot\cos\theta\]
			\[u^1=r\cdot\cos\varphi\cdot\cos\theta-r\cdot\sin\varphi\cdot\sin\theta-r\cdot\cos\theta\]
			
			\[u^2=r\cdot\sin(\varphi+\theta)-r\cdot\sin\theta\]
			\[u^2=r\cdot\sin\varphi\cdot\cos\theta+r\cdot\sin\theta\cdot\cos\varphi-r\cdot\sin\theta\]
			
			Substituindo as coordenadas de referência nas coordenadas deformadas:
			
			\[u^1=x^{r1}\cdot\cos\varphi-x^{r2}\cdot\sin\varphi-x^{r1}\]
			\[u^2=x^{r1}\cdot\sin\varphi+x^{r2}\cdot\cos\varphi-x^{r2}\]
			\[u^3=0\]
			
			Como sabemos que $\utilde{\mathbf{x}}=\utilde{\mathbf{x^r}}+\utilde{\mathbf{u}}$, temos:
			
			\[
				\begin{cases}
					x^1=x^{r1}\cdot\cos\varphi-x^{r2}\cdot\sin\varphi \\ x^2=x^{r1}\cdot\sin\varphi+x^{r2}\cdot\cos\varphi \\ x^3=x^{r3}
				\end{cases}
			\]
			
			Podemos escrever também na forma matricial:
			
			\[
				\begin{Bmatrix}
					x^1 \\ x^2 \\ x^3
				\end{Bmatrix}
				=
				\begin{bmatrix}
				
					\cos\varphi & -\sin\varphi & 0 \\
					\sin\varphi & \cos\varphi & 0 \\
					0 & 0 & 1
				\end{bmatrix}
				\begin{Bmatrix}
					x^{r1} \\ x^{r2} \\ x^{r3}
				\end{Bmatrix}							
			\]
	\end{enumerate}
		
	\subsection{Deformação Normal e por Cisalhamento}
	
	//Inserir imagem
	
	O assunto agora são fibras; como as fibras sofrem deformação. Seja $d\utilde{\mathbf{x}}^r$ o vetor infinitesimal que representa uma fibra a partir do ponto $\utilde{\mathbf{x^r}}$; $d\utilde{\mathbf{x}}$ o vetor infinitesimal da mesma fibra agora na configuração deformada, partindo do ponto $\utilde{\mathbf{x}}$.
	Para que o vetor $d\utilde{\mathbf{x}}^r$ deforme e se transforme em $d\utilde{\mathbf{x}}$, deve ocorrer uma transformação que depende de $\utilde{\mathbf{x}}^r+d\utilde{\mathbf{x}}^r$, \textit{i.e.}, uma transformação $\utilde{\mathbf{u}}(\utilde{\mathbf{x}}^r+d\utilde{\mathbf{x}}^r)$.
	
	Algumas relações podem ser estabelecidas a partir da imagem acima, sendo:
	\[\utilde{\mathbf{u}}(\utilde{\mathbf{x}}^r)=d\utilde{\mathbf{x}}^r+\utilde{\mathbf{u}}(\utilde{\mathbf{x}}^r+d\utilde{\mathbf{x}}^r)\]
	
	E a fibra na configuração deformada:	
	\[d\utilde{\mathbf{x}}=d\utilde{\mathbf{x}}^r+\utilde{\mathbf{u}}(\utilde{\mathbf{x}}^r+d\utilde{\mathbf{x}}^r)-\utilde{\mathbf{u}}(\utilde{\mathbf{x}}^r)\]
	
	A equação acima na forma de componentes:
	\[dx^i=dx^{ri}+u^i(x^{r1}+dx^{r1},\;x^{r2}+dx^{r2},\;x^{r3}+dx^{r3})-u^i(x^{r1},\;x^{r2},\;x^{r3})\]
	
	Lembrando de cálculo com múltiplas variáveis:
	\[u^i(x^{r1}+dx^{r1},\;x^{r2}+dx^{r2},\;x^{r3}+dx^{r3})-u^i(x^{r1},\;x^{r2},\;x^{r3})=\frac{\partial u^i}{\partial x^{r1}}dx^{r1}+\frac{\partial u^i}{\partial x^{r2}}dx^{r2}+\frac{\partial u^i}{\partial x^{r3}}dx^{r3}\]
	
	Para $i=1,\;2$ e $3$. Essa iteração diz que cada coordenada $i$ depende de acontecimentos nas $3$ dimensões do espaço euclidiano.
	
	Reescrevendo na forma matricial:
	\[
	\underbrace{
		\begin{Bmatrix}
			dx^1 \\ dx^2 \\ dx^3
		\end{Bmatrix}
		}_{\displaystyle d\utilde{\mathbf{x}}}
		=
		\underbrace{
		\begin{Bmatrix}
			dx^{r1} \\ dx^{r2} \\ dx^{r3}
		\end{Bmatrix}
		}_{\displaystyle d\utilde{\mathbf{x}}^r}
		+
		\underbrace{
		\begin{bmatrix}
			\frac{\partial u^1}{\partial x^{r1}} & \frac{\partial u^1}{\partial x^{r2}} & \frac{\partial u^1}{\partial x^{r3}} \\
			\frac{\partial u^2}{\partial x^{r1}} & \frac{\partial u^2}{\partial x^{r2}} & \frac{\partial u^2}{\partial x^{r3}} \\
			\frac{\partial u^3}{\partial x^{r1}} & \frac{\partial u^3}{\partial x^{r2}} & \frac{\partial u^3}{\partial x^{r3}}
		\end{bmatrix}
		}_{\displaystyle\nabla\utilde{\mathbf{u}}=\underline{\mathbf{L}}}
		\underbrace{
		\begin{Bmatrix}
			dx^{r1} \\ dx^{r2} \\ dx^{r3}
		\end{Bmatrix}
		}_{\displaystyle d\utilde{\mathbf{x}}^r}
	\]
	
	Onde $\nabla\utilde{\mathbf{u}}$ é o \textbf{gradiente dos deslocamentos}.
	
	Logo,
	\[d\utilde{\mathbf{x}}=d\utilde{\mathbf{x}}^r+\nabla\utilde{\mathbf{u}}\;d\utilde{\mathbf{x}}^r\]
	
	Colocando $d\utilde{\mathbf{x}}^r$ em evidência, temos:
	\[d\utilde{\mathbf{x}}=\underbrace{(\underline{\mathbf{I}}+\nabla\utilde{\mathbf{u}})}_{\displaystyle \underline{\mathbf{F}}}d\utilde{\mathbf{x}}^r\]
	
	Onde $\underline{\mathbf{F}}$ é o \textbf{gradiente das deformações}.
	
	Portanto, temos:
	\begin{equation}
	d\utilde{\mathbf{x}}=\underline{\mathbf{F}}\;d\utilde{\mathbf{x}}^r
	\end{equation}
	
	Ou seja, a fibra deformada agora pode ser obtida a partir do gradiente dos deslocamentos ($\nabla\utilde{\mathbf{u}}$) e também a partir do gradiente das deformações ($\underline{\mathbf{F}}$).
	
	O gradiente das deformações pode ser reescrito como:
	\[
	\underline{\mathbf{F}}=
	\begin{bmatrix}
		1+\frac{\partial u^1}{\partial x^{r1}} & \frac{\partial u^1}{\partial x^{r2}} & \frac{\partial u^1}{\partial x^{r3}} \\
		\frac{\partial u^2}{\partial x^{r1}} & 1+\frac{\partial u^2}{\partial x^{r2}} & \frac{\partial u^2}{\partial x^{r3}} \\
		\frac{\partial u^3}{\partial x^{r1}} & \frac{\partial u^3}{\partial x^{r2}} & 1+\frac{\partial u^3}{\partial x^{r3}}
	\end{bmatrix}
	\]
	
	Ou ainda, usando notação indicial de Einstein:
	\[F_{ij}=\nabla u_{ij}+\delta_{ij}\]
	
	Onde $\delta_{ij}$ é o delta de Kronecker, \textit{i.e.}: $\delta_{ij}=\begin{cases} 1, & \text{se } i=j \\ 0, & \text{se } i\neq j \end{cases}$
	
	E em notação indicial, temos:
	\[\nabla u_{ij}=\frac{\partial u^i}{\partial x^{rj}}\]
	
	Agora, expressemos:
	\[\utilde{\mathbf{u}}=\utilde{\mathbf{x}}-\utilde{\mathbf{x}}^r\implies \utilde{\mathbf{x}}=\utilde{\mathbf{u}}+\utilde{\mathbf{x}}^r\]
	\[x^i=u^i+x^{ri}\]
	
	Derivando ambos os lados da notação indicial acima em relação a $\utilde{\mathbf{x}}^r$, temos:
	\[\underbrace{\frac{\partial x^i}{\partial x^{rj}}}_{\displaystyle\underline{\mathbf{F}}}=\underbrace{\frac{\partial u^i}{\partial x^{rj}}}_{\displaystyle\nabla u_{ij}}+\underbrace{\frac{\partial x^{ri}}{\partial x^{rj}}}_{\displaystyle\delta_{ij}}\]
	
	Ou seja, agora o gradiente das deformações está em função da transformação e sua forma matricial é:
	\[
	\underline{\mathbf{F}}=
	\begin{bmatrix}
		\frac{\partial x^1}{\partial x^{r1}} & \frac{\partial x^1}{\partial x^{r2}} & \frac{\partial x^1}{\partial x^{r3}} \\
		\frac{\partial x^2}{\partial x^{r1}} & \frac{\partial x^2}{\partial x^{r2}} & \frac{\partial x^2}{\partial x^{r3}} \\
		\frac{\partial x^3}{\partial x^{r1}} & \frac{\partial x^3}{\partial x^{r2}} & \frac{\partial x^3}{\partial x^{r3}}
	\end{bmatrix}	
	\]
	
	Lembrete sobre operadores lineares:
	
	$\underline{\mathbf{F}}$ e $\nabla\utilde{\mathbf{u}}$  são operadores lineares de $E\rightarrow E$. Seja $\mathbf{\underline{T}}$ um operador linear:
	\[\mathbf{\underline{T}}(\alpha\utilde{\mathbf{x}}+\beta\utilde{\mathbf{y}})=\alpha\mathbf{\underline{T}}\utilde{\mathbf{x}}+\beta\mathbf{\underline{T}}\utilde{\mathbf{y}},\;\forall\;\mathbf{\utilde{\mathbf{x}}}\text{ e }\mathbf{\utilde{\mathbf{y}}}\in\text{$E$}\]
	
	//Inserir imagem
	
	\[
		\begin{Bmatrix}
		b_1 \\ b_2 \\ b_3
		\end{Bmatrix}
		=
		\underbrace{
		\begin{bmatrix}
		T_{11} & T_{12} & T_{13} \\
		T_{21} & T_{22} & T_{23} \\
		T_{31} & T_{32} & T_{33}
		\end{bmatrix}
		}_{\displaystyle\mathbf{\underline{T}}}
		\begin{Bmatrix}
		a_1 \\ a_2 \\ a_3
		\end{Bmatrix}
	\]
	
	Obs: Para o curso, um operador linear tem a mesma função de um tensor (simplificando).
	
	\subsection{Tensor das Deformações de Green-Lagrange}
	
	//Inserir imagem
	
	Onde $\hat{\utilde{\mathbf{m}}}^r$ é um versor, $\displaystyle\frac{d\utilde{\mathbf{x}}^r}{||d\utilde{\mathbf{x}}^r||}=1$, $ds^r$ é o comprimento infinitesimal de uma fibra na configuração de referência na direção de $\hat{\utilde{\mathbf{m}}}^r$ e $ds$ é o comprimento dessa mesma fibra na configuração deformada. Os comprimentos são definidos como:
	\[ds^r=||d\utilde{\mathbf{x}}^r||=(d\utilde{\mathbf{x}}^r\cdot d\utilde{\mathbf{x}}^r)^{\frac{1}{2}}\]
	\[ds=||d\utilde{\mathbf{x}}||=(d\utilde{\mathbf{x}}\cdot d\utilde{\mathbf{x}})^{\frac{1}{2}}\]
	
	Lembrando sobre operadores lineares: Seja $\underline{\mathbf{A}}$ um operador linear ou tensor, temos
	\[\utilde{\mathbf{a}}\cdot\underline{\mathbf{A}}\utilde{\mathbf{b}}=\utilde{\mathbf{b}}\cdot\underline{\mathbf{A}}^{\text{T}}\utilde{\mathbf{a}},\;\forall\;\mathbf{\utilde{\mathbf{a}}}\text{ e }\mathbf{\utilde{\mathbf{b}}}\in\text{$E$}\]
	
	Quando $\underline{\mathbf{A}}=\underline{\mathbf{A}}^{\text{T}}$, $\underline{\mathbf{A}}$ é simétrico.
	
	\textbf{Exemplo}: A \textit{interpolação quadrática} ou \textit{alongamento quadrático} é definida como:
	\[\varepsilon_q=\frac{1}{2}\frac{(ds)^2-(ds^r)^2}{(ds^r)^2}\]
	
	Sabendo que,
	\[ds^2=d\utilde{\mathbf{x}}\cdot d\utilde{\mathbf{x}}=\underbrace{\underline{\mathbf{F}}d\utilde{\mathbf{x}}^r}_{\displaystyle\utilde{\mathbf{v}}}\cdot\underline{\mathbf{F}}d\utilde{\mathbf{x}}^r\]
	\[ds^2=\utilde{\mathbf{v}}\cdot\underline{\mathbf{F}}d\utilde{\mathbf{x}}^r\]
	\[ds^2=d\utilde{\mathbf{x}}^r\cdot\underline{\mathbf{F}}^{\text{T}}\utilde{\mathbf{v}}\]
	\[ds^2=d\utilde{\mathbf{x}}^r\cdot\underline{\mathbf{F}}^{\text{T}}\underline{\mathbf{F}}d\utilde{\mathbf{x}}^r\]
	
	Logo,
	\[\varepsilon_q=\frac{1}{2}\frac{(d\utilde{\mathbf{x}}^r\cdot\underline{\mathbf{F}}^{\text{T}}\underline{\mathbf{F}}d\utilde{\mathbf{x}}^r)-(d\utilde{\mathbf{x}}^r\cdot d\utilde{\mathbf{x}}^r)}{d\utilde{\mathbf{x}}^r\cdot d\utilde{\mathbf{x}}^r}\]
	
	Lembrando que,
	\[d\utilde{\mathbf{x}}^r=||d\utilde{\mathbf{x}}^r||\hat{\utilde{\mathbf{m}}}^r=ds^r\hat{\utilde{\mathbf{m}}}^r\]
	
	Então,
	\[\varepsilon_q=\frac{1}{2}\frac{(ds^r\hat{\utilde{\mathbf{m}}}^r\cdot\underline{\mathbf{F}}^{\text{T}}\underline{\mathbf{F}}d\utilde{\mathbf{x}}^r)-(ds^r\hat{\utilde{\mathbf{m}}}^r\cdot ds^r\hat{\utilde{\mathbf{m}}}^r)}{ds^r\hat{\utilde{\mathbf{m}}}^r\cdot ds^r\hat{\utilde{\mathbf{m}}}^r}\]
	\[\varepsilon_q=\frac{1}{2}\frac{ds^r\hat{\utilde{\mathbf{m}}}^r\cdot(\underline{\mathbf{F}}^{\text{T}}\underline{\mathbf{F}}-\underline{\mathbf{I}})ds^r\hat{\utilde{\mathbf{m}}}^r}{ds^r\hat{\utilde{\mathbf{m}}}^r\cdot ds^r\hat{\utilde{\mathbf{m}}}^r}\]
	\[\varepsilon_q=\frac{1}{2}[\hat{\utilde{\mathbf{m}}}^r\cdot(\underline{\mathbf{F}}^{\text{T}}\underline{\mathbf{F}}-\underline{\mathbf{I}})\hat{\utilde{\mathbf{m}}}^r]\]
	\[\varepsilon_q=\hat{\utilde{\mathbf{m}}}^r\cdot\underbrace{\frac{1}{2}(\underline{\mathbf{F}}^{\text{T}}\underline{\mathbf{F}}-\underline{\mathbf{I}})}_{\displaystyle\underline{\mathbf{E}}}\hat{\utilde{\mathbf{m}}}^r\]
	
	\[\underline{\mathbf{E}}=\frac{1}{2}(\underline{\mathbf{F}}^{\text{T}}\underline{\mathbf{F}}-\underline{\mathbf{I}})\]
	\[\varepsilon_q(\hat{\utilde{\mathbf{m}}}^r)=\hat{\utilde{\mathbf{m}}}^r\cdot\underline{\mathbf{E}}\hat{\utilde{\mathbf{m}}}^r\]
	
	Onde $\underline{\mathbf{E}}$ é o Tensor das Deformações de Green-Lagrange e vale para qualquer magnitude de deslocamento. Algumas literaturas chamam de \textit{finite displacement} os deslocamentos de qualquer magnitude.
	
	Agora na notação de Einstein:
	\[E_{ij}=\frac{1}{2}\left(\frac{\partial x^k}{\partial x^{ri}}\frac{\partial x^k}{\partial x^{rj}}-\delta_{ij}\right)\]
	
	Expressando os componentes de $\underline{\mathbf{E}}$ a partir do campo de deslocamentos:
	\[\underline{\mathbf{E}}=\frac{1}{2}[(\nabla\utilde{\mathbf{u}}^{\text{T}}+\underline{\mathbf{I}})(\nabla\utilde{\mathbf{u}}+\underline{\mathbf{I}})-\underline{\mathbf{I}}]\]
	\[\underline{\mathbf{E}}=\frac{1}{2}(\nabla\utilde{\mathbf{u}}^{\text{T}}\nabla\utilde{\mathbf{u}}+\nabla\utilde{\mathbf{u}}^{\text{T}}+\nabla\utilde{\mathbf{u}}+\underline{\mathbf{I}}-\underline{\mathbf{I}})\]
	\[\underline{\mathbf{E}}=\frac{1}{2}(\nabla\utilde{\mathbf{u}}+\nabla\utilde{\mathbf{u}}^{\text{T}}+\nabla\utilde{\mathbf{u}}^{\text{T}}\nabla\utilde{\mathbf{u}})\]
	
	Na notação de Einstein:
	\[E_{ij}=\frac{1}{2}\left(\frac{\partial u^i}{\partial x^{rj}}+\frac{\partial u^j}{\partial x^{ri}}+\underbrace{\frac{\partial u^k}{\partial x^{ri}}\frac{\partial u^k}{\partial x^{rj}}}_{*}\right)\]
	
	Sendo $*$:
	\[\frac{\partial u^k}{\partial x^{ri}}\frac{\partial u^k}{\partial x^{rj}}=\frac{\partial u^1}{\partial x^{ri}}\frac{\partial u^1}{\partial x^{rj}}+\frac{\partial u^2}{\partial x^{ri}}\frac{\partial u^2}{\partial x^{rj}}+\frac{\partial u^3}{\partial x^{ri}}\frac{\partial u^3}{\partial x^{rj}}\]
	
	Acima, $E_{ij}$ denota equações de compatibilidade; uma relação deslocamentos-deformações.
	
	\subsection{Medidas de deformação/alongamento}
	
	\begin{itemize}
		\item Estiramento ou \textit{stretch}:
			\[\lambda=\frac{ds}{ds^r}\]
		\item Alongamento linear:
			\[\varepsilon_l=\frac{ds-ds^r}{ds^r}\]
		\item Alongamento quadrático:
			\[\varepsilon_q=\frac{1}{2}\frac{(ds)^2-(ds^r)^2}{(ds^r)^2}=\frac{1}{2}\left[\left(\frac{ds}{ds^r}\right)^2-1\right]=\frac{1}{2}(\lambda^2-1)\]
		\item Alongamento logarítmico ou de Henry:
			\[\varepsilon_n=\ln \left(\frac{ds}{ds^r}\right)\]
		\item Alongamento hiperbólico ou de Reiner:
			\[\varepsilon_h=\frac{ds^r}{ds}\]
		\item Alongamento hiperbólico quadrático ou de Almansi:
			\[\varepsilon_{hq}=\frac{1}{2}\frac{(ds^r)^2-(ds)^2}{(ds)^2}\]
	\end{itemize}
	
	\subsubsection{Relação entre alongamentos quadráticos e lineares}
	
	Para o alongamento quadrático:
	\[\varepsilon_q=\frac{1}{2}(\lambda^2-1)\]
	\[\lambda^2=1+2\varepsilon_q\]
	\[\lambda=\sqrt{1+2\varepsilon_q}\]
	\[\lambda(\hat{\utilde{\mathbf{m}}}^r)=\sqrt{1+2(\hat{\utilde{\mathbf{m}}}^r\cdot\underline{\mathbf{E}}\hat{\utilde{\mathbf{m}}}^r)}\]
	
	Para o alongamento linear:
	\[\varepsilon_l=\lambda-1\]
	\[\varepsilon_l(\hat{\utilde{\mathbf{m}}}^r)=\sqrt{1+2(\hat{\utilde{\mathbf{m}}}^r\cdot\underline{\mathbf{E}}\hat{\utilde{\mathbf{m}}}^r)}-1\]
	\[\varepsilon_l(\hat{\utilde{\mathbf{m}}}^r)=\sqrt{1+2\varepsilon_q}-1\]
	
	\subsection{Variação do ângulo entre fibras (distorção)}
	
	//Inserir imagem
	
	\begin{itemize}
		\item $d\utilde{\mathbf{a}}^r$ e $d\utilde{\mathbf{b}}^r$ são inicialmente ortogonais;
		\item $\underline{\mathbf{F}}d\utilde{\mathbf{a}}^r$ tem a mesma direção de $d\utilde{\mathbf{a}}$ (hipótese);
		\item $\hat{\utilde{\mathbf{a}}}$ e $\hat{\utilde{\mathbf{b}}}$ são versores.
	\end{itemize}
	
	Lembrando que,
	\[d\utilde{\mathbf{a}}=\underline{\mathbf{F}}d\utilde{\mathbf{a}}^r\text{ e }d\utilde{\mathbf{b}}=\underline{\mathbf{F}}d\utilde{\mathbf{b}}^r\]
	
	A fim de encontrar novas relações, podemos definir o produto escalar,
	\[d\utilde{\mathbf{a}}\cdot d\utilde{\mathbf{b}}=\underline{\mathbf{F}}d\utilde{\mathbf{a}}^r\cdot\underline{\mathbf{F}}d\utilde{\mathbf{b}}^r\]
	\[d\utilde{\mathbf{a}}\cdot d\utilde{\mathbf{b}}=||\underline{\mathbf{F}}d\utilde{\mathbf{a}}^r||\;|| \underline{\mathbf{F}}d\utilde{\mathbf{b}}^r ||\cos\theta\]
	
	Logo,
	\begin{equation}\label{eq-variacao-angulo-fibras}
		\cos\theta=\sin\gamma=\frac{\underline{\mathbf{F}}d\utilde{\mathbf{a}}^r\cdot\underline{\mathbf{F}}d\utilde{\mathbf{b}}^r}{||\underline{\mathbf{F}}d\utilde{\mathbf{a}}^r||\;|| \underline{\mathbf{F}}d\utilde{\mathbf{b}}^r ||}
	\end{equation}
	
	E,
	\[d\utilde{\mathbf{a}}^r=d\mathbf{a}^r\hat{\utilde{\mathbf{a}}}^r\]
	\[d\utilde{\mathbf{b}}^r=d\mathbf{b}^r\hat{\utilde{\mathbf{b}}}^r\]
	\[||\underline{\mathbf{F}}d\utilde{\mathbf{a}}^r||=\sqrt{\underline{\mathbf{F}}d\utilde{\mathbf{a}}^r\cdot\underline{\mathbf{F}}d\utilde{\mathbf{a}}^r}\]
	\[||\underline{\mathbf{F}}d\utilde{\mathbf{b}}^r||=\sqrt{\underline{\mathbf{F}}d\utilde{\mathbf{b}}^r\cdot\underline{\mathbf{F}}d\utilde{\mathbf{b}}^r}\]
	
	Substituindo as 4 relações acima na Equação \eqref{eq-variacao-angulo-fibras},
	\[\sin\gamma=\frac{d\utilde{\mathbf{b}}^r\cdot\underline{\mathbf{F}}^{\text{T}}\underline{\mathbf{F}}d\utilde{\mathbf{a}}^r}{\sqrt{d\utilde{\mathbf{a}}^r\cdot\underline{\mathbf{F}}^{\text{T}}\underline{\mathbf{F}}d\utilde{\mathbf{a}}^r}\sqrt{d\utilde{\mathbf{b}}^r\cdot\underline{\mathbf{F}}^{\text{T}}\underline{\mathbf{F}}d\utilde{\mathbf{b}}^r}}\]
	
	\[\sin\gamma=\frac{(d\mathbf{b}^r\hat{\utilde{\mathbf{b}}}^r)\cdot[\underline{\mathbf{F}}^{\text{T}}\underline{\mathbf{F}}(d\mathbf{a}^r\hat{\utilde{\mathbf{a}}}^r)]}{\sqrt{(d\mathbf{a}^r\hat{\utilde{\mathbf{a}}}^r)\cdot[\underline{\mathbf{F}}^{\text{T}}\underline{\mathbf{F}}(d\mathbf{a}^r\hat{\utilde{\mathbf{a}}}^r)]}\sqrt{(d\mathbf{b}^r\hat{\utilde{\mathbf{b}}}^r)\cdot[\underline{\mathbf{F}}^{\text{T}}\underline{\mathbf{F}}(d\mathbf{b}^r\hat{\utilde{\mathbf{b}}}^r)]}}\]
	
	\[\sin\gamma=\frac{(d\mathbf{b}^rd\mathbf{a}^r)[\hat{\utilde{\mathbf{b}}}^r\cdot\underline{\mathbf{F}}^{\text{T}}\underline{\mathbf{F}}\hat{\utilde{\mathbf{a}}}^r]}{(d\mathbf{b}^rd\mathbf{a}^r)\sqrt{\hat{\utilde{\mathbf{a}}}^r\cdot\underline{\mathbf{F}}^{\text{T}}\underline{\mathbf{F}}\hat{\utilde{\mathbf{a}}}^r}\sqrt{\hat{\utilde{\mathbf{b}}}^r\cdot\underline{\mathbf{F}}^{\text{T}}\underline{\mathbf{F}}\hat{\utilde{\mathbf{b}}}^r}}\]
	
	\begin{equation}\label{eq-distorcao-fibra-1}
		\sin\gamma=\frac{\hat{\utilde{\mathbf{b}}}^r\cdot\underline{\mathbf{F}}^{\text{T}}\underline{\mathbf{F}}\hat{\utilde{\mathbf{a}}}^r}{\sqrt{\hat{\utilde{\mathbf{a}}}^r\cdot\underline{\mathbf{F}}^{\text{T}}\underline{\mathbf{F}}\hat{\utilde{\mathbf{a}}}^r}\sqrt{\hat{\utilde{\mathbf{b}}}^r\cdot\underline{\mathbf{F}}^{\text{T}}\underline{\mathbf{F}}\hat{\utilde{\mathbf{b}}}^r}}
	\end{equation}		

	Lembrando que,
	\[\underline{\mathbf{E}}=\frac{1}{2}(\underline{\mathbf{F}}^{\text{T}}\underline{\mathbf{F}}-\underline{\mathbf{I}})\]
	\[\underline{\mathbf{F}}^{\text{T}}\underline{\mathbf{F}}=2\underline{\mathbf{E}}+\underline{\mathbf{I}}\]
	
	Substituindo a expressão acima na Equação \eqref{eq-distorcao-fibra-1},
	\[\sin\gamma=\frac{\hat{\utilde{\mathbf{b}}}^r\cdot(2\underline{\mathbf{E}}+\underline{\mathbf{I}})\hat{\utilde{\mathbf{a}}}^r}{\sqrt{\hat{\utilde{\mathbf{a}}}^r\cdot(2\underline{\mathbf{E}}+\underline{\mathbf{I}})\hat{\utilde{\mathbf{a}}}^r}\sqrt{\hat{\utilde{\mathbf{b}}}^r\cdot(2\underline{\mathbf{E}}+\underline{\mathbf{I}})\hat{\utilde{\mathbf{b}}}^r}}\]
	\[(2\underline{\mathbf{E}}+\underline{\mathbf{I}})\hat{\utilde{\mathbf{a}}}^r=2\underline{\mathbf{E}}\hat{\utilde{\mathbf{a}}}^r+\underline{\mathbf{I}}\hat{\utilde{\mathbf{a}}}^r=2\underline{\mathbf{E}}\hat{\utilde{\mathbf{a}}}^r+\hat{\utilde{\mathbf{a}}}^r\]
	
	\[\sin\gamma=\frac{\hat{\utilde{\mathbf{b}}}^r\cdot(2\underline{\mathbf{E}}\hat{\utilde{\mathbf{a}}}^r+\hat{\utilde{\mathbf{a}}}^r)}{\sqrt{\hat{\utilde{\mathbf{a}}}^r\cdot(2\underline{\mathbf{E}}\hat{\utilde{\mathbf{a}}}^r+\hat{\utilde{\mathbf{a}}}^r)}\sqrt{\hat{\utilde{\mathbf{b}}}^r\cdot(2\underline{\mathbf{E}}\hat{\utilde{\mathbf{b}}}^r+\hat{\utilde{\mathbf{b}}}^r)}}\]
	
	\[\sin\gamma=\frac{\hat{\utilde{\mathbf{b}}}^r\cdot2\underline{\mathbf{E}}\hat{\utilde{\mathbf{a}}}^r+\hat{\utilde{\mathbf{b}}}^r\cdot\hat{\utilde{\mathbf{a}}}^r}{\displaystyle\sqrt{\hat{\utilde{\mathbf{a}}}^r\cdot2\underline{\mathbf{E}}\hat{\utilde{\mathbf{a}}}^r+\hat{\utilde{\mathbf{a}}}^r\cdot\hat{\utilde{\mathbf{a}}}^r}\sqrt{\hat{\utilde{\mathbf{b}}}^r\cdot2\underline{\mathbf{E}}\hat{\utilde{\mathbf{b}}}^r+\hat{\utilde{\mathbf{b}}}^r\cdot\hat{\utilde{\mathbf{b}}}^r}}\]
	
	Por fim,
	\begin{equation}\label{eq-distorcao-fibra-2}
		\sin\gamma(\hat{\utilde{\mathbf{a}}}^r,\hat{\utilde{\mathbf{b}}}^r)=\frac{2(\hat{\utilde{\mathbf{b}}}^r\cdot\underline{\mathbf{E}}\hat{\utilde{\mathbf{a}}}^r)}{\displaystyle\sqrt{2(\hat{\utilde{\mathbf{a}}}^r\cdot\underline{\mathbf{E}}\hat{\utilde{\mathbf{a}}}^r)+1}\sqrt{2(\hat{\utilde{\mathbf{b}}}^r\cdot\underline{\mathbf{E}}\hat{\utilde{\mathbf{b}}}^r)+1}}
	\end{equation}
	
	Onde $\gamma$ é a distorção, \textit{i.e.}, a variação de ângulo entre duas fibras inicialmente ortogonais.
	
	\subsection{Interpretação de $\underline{\mathbf{E}}$}
	
	Partindo-se de uma base ortonormal ($\utilde{e_1}$, $\utilde{e_2}$, $\utilde{e_3}$), o Tensor das Deformações de Green-Lagrange é definido como,
	\[
	\underline{\mathbf{E}}
	=
	\begin{bmatrix}
	E_{11} & E_{12} & E_{13} \\
	E_{21} & E_{22} & E_{23} \\
	E_{31} & E_{32} & E_{33}
	\end{bmatrix}
	\]
	
	Lembrando do alongamento quadrático, por exemplo, de uma fibra alinhada na direção de $\utilde{\mathbf{\hat{e}_1}}$,
	\[\varepsilon_q(\utilde{\mathbf{\hat{e}_1}})=\utilde{\mathbf{\hat{e}_1}}\cdot\underline{\mathbf{E}}\utilde{\mathbf{\hat{e}_1}}\]
	\[
	\varepsilon_q(\utilde{\mathbf{\hat{e}_1}})
	=
	\begin{Bmatrix}
	1 & 0 & 0
	\end{Bmatrix}
	\cdot
	\begin{bmatrix}
	E_{11} & E_{12} & E_{13} \\
	E_{21} & E_{22} & E_{23} \\
	E_{31} & E_{32} & E_{33}
	\end{bmatrix}
	\begin{Bmatrix}
	1 \\ 0 \\ 0
	\end{Bmatrix}
	\]
	\[
	\varepsilon_q(\utilde{\mathbf{\hat{e}_1}})
	=
	\begin{Bmatrix}
	1 & 0 & 0
	\end{Bmatrix}
	\cdot
	\begin{Bmatrix}
	E_{11} \\ E_{22} \\ E_{33}
	\end{Bmatrix}
	=
	E_{11}
	\]
	
	Da mesma forma, uma fibra na direção de $\utilde{\mathbf{\hat{e}_2}}$ ou $\utilde{\mathbf{\hat{e}_3}}$,
	\[\varepsilon_q(\utilde{\mathbf{\hat{e}_2}})=E_{22}\]
	\[\varepsilon_q(\utilde{\mathbf{\hat{e}_3}})=E_{33}\]
	
	Ou seja, a diagonal de $\underline{\mathbf{E}}$ guarda o alongamento quadrático. E quanto à distorção? Escolhendo dois vetores da base ortonormal quaisqueres, por exemplo, $\utilde{\mathbf{\hat{e}_1}}$ e $\utilde{\mathbf{\hat{e}_2}}$ e lembrando da Equação \eqref{eq-distorcao-fibra-2},
	\begin{equation}\label{eq-distorcao-fibra-3}
		\sin\gamma(\utilde{\mathbf{\hat{e}_1}},\utilde{\mathbf{\hat{e}_2}})=\frac{2(\utilde{\mathbf{\hat{e}_2}}\cdot\underline{\mathbf{E}}\utilde{\mathbf{\hat{e}_1}})}{\displaystyle\sqrt{2(\utilde{\mathbf{\hat{e}_1}}\cdot\underline{\mathbf{E}}\utilde{\mathbf{\hat{e}_1}})+1}\sqrt{2(\utilde{\mathbf{\hat{e}_2}}\cdot\underline{\mathbf{E}}\utilde{\mathbf{\hat{e}_2}})+1}}
	\end{equation}
	
	Desenvolvendo o produto escalar no numerador da expressão acima na equação do alongamento quadrático,
	\[\varepsilon_q=\utilde{\mathbf{\hat{e}_2}}\cdot\underline{\mathbf{E}}\utilde{\mathbf{\hat{e}_1}}\]
	\[
	\varepsilon_q
	=
	\begin{Bmatrix}
	0 & 1 & 0
	\end{Bmatrix}
	\cdot
	\begin{bmatrix}
	E_{11} & E_{12} & E_{13} \\
	E_{21} & E_{22} & E_{23} \\
	E_{31} & E_{32} & E_{33}
	\end{bmatrix}
	\begin{Bmatrix}
	1 \\ 0 \\ 0
	\end{Bmatrix}
	\]
	\[
	\varepsilon_q
	=
	\begin{Bmatrix}
	0 & 1 & 0
	\end{Bmatrix}
	\cdot
	\begin{Bmatrix}
	E_{11} \\ E_{21} \\ E_{31}
	\end{Bmatrix}
	=
	E_{21}	
	\]
	
	Agora, invertendo os versores escolhidos para a Equação \eqref{eq-distorcao-fibra-2} e como o denominador continuará o mesmo, só o numerador sofrerá alteração. Desenvolvendo-o da mesma forma na equação do alongamento quadrático,
	\[\varepsilon_q=\utilde{\mathbf{\hat{e}_1}}\cdot\underline{\mathbf{E}}\utilde{\mathbf{\hat{e}_2}}\]
	\[
	\varepsilon_q
	=
	\begin{Bmatrix}
	1 & 0 & 0
	\end{Bmatrix}
	\cdot
	\begin{bmatrix}
	E_{11} & E_{12} & E_{13} \\
	E_{21} & E_{22} & E_{23} \\
	E_{31} & E_{32} & E_{33}
	\end{bmatrix}
	\begin{Bmatrix}
	0 \\ 1 \\ 0
	\end{Bmatrix}
	\]
	\[
	\varepsilon_q
	=
	\begin{Bmatrix}
	1 & 0 & 0
	\end{Bmatrix}
	\cdot
	\begin{Bmatrix}
	E_{12} \\ E_{22} \\ E_{32}
	\end{Bmatrix}
	=
	E_{12}
	\]
	
	Ou seja, como a distorção entre esses versores deve ser igual, apenas invertendo os versores utilizados pode-se chegar a conclusão de que $E_{21}=E_{12}$. Isso significa que $\underline{\mathbf{E}}$ é simétrico.
	
	Pode-se reescrever a Equação \eqref{eq-distorcao-fibra-3} como,
	\[\sin\gamma(\utilde{\mathbf{\hat{e}_1}},\utilde{\mathbf{\hat{e}_2}})=\frac{2E_{21}}{\displaystyle\sqrt{2E_{11}+1}\sqrt{2E_{22}+1}}\]
	
	\subsection{Tensor das Deformações de Cauchy-Green (da direita)}
	
	O Tensor das Deformações de Cauchy-Green ($\underline{\mathbf{C}}$) serve para grandes deformações e é definido como,
	\begin{equation}\label{eq-def-cauchy-green}
		\underline{\mathbf{C}}=\underline{\mathbf{F}}^{\text{T}}\underline{\mathbf{F}}
	\end{equation}
	
	Podendo ser reescrito na notação de Einstein como,
	\[C_{ij}=F_{ik}^{\text{T}}F_{kj}\]
	\[C_{ij}=F_{ki}F_{kj}\]
	
	Na forma diferencial,
	\[C_{ij}=\frac{\partial \utilde{\mathbf{x}}^k}{\partial x^{ri}}\frac{\partial \utilde{\mathbf{x}}^k}{\partial x^{rj}}\]
	
	\subsubsection{Relações entre alongamentos quadráticos e lineares}
	
	O alongamento quadrático,
	\[\lambda=\sqrt{2\varepsilon_q+1}\]
	
	Elevando ambos os lados ao quadrado,
	\[\lambda^2=2\varepsilon_q+1=2(\hat{\utilde{\mathbf{m}}}^r\cdot\underline{\mathbf{E}}\hat{\utilde{\mathbf{m}}}^r)+1\]
	
	Como $\underline{\mathbf{C}}=\underline{\mathbf{F}}^{\text{T}}\underline{\mathbf{F}}$, $\underline{\mathbf{E}}$ pode ser reescrito como,
	\begin{equation}\label{eq-green-lagrange-em-funcao-de-cauchy-green}
		\underline{\mathbf{E}}=\frac{1}{2}(\underline{\mathbf{C}}-\underline{\mathbf{I}})
	\end{equation}
	
	Pode-se reescrever $\lambda^2$ como,
	\[\lambda^2=2\left\{\hat{\utilde{\mathbf{m}}}^r\cdot\left[\frac{1}{2}(\underline{\mathbf{C}}-\underline{\mathbf{I}})\right]\hat{\utilde{\mathbf{m}}}^r\right\}+1\]
	\[\lambda^2=\hat{\utilde{\mathbf{m}}}^r\cdot\underline{\mathbf{C}}\hat{\utilde{\mathbf{m}}}^r-\hat{\utilde{\mathbf{m}}}^r\cdot\underline{\mathbf{I}}\hat{\utilde{\mathbf{m}}}^r+1\]
	\[\lambda^2=\hat{\utilde{\mathbf{m}}}^r\cdot\underline{\mathbf{C}}\hat{\utilde{\mathbf{m}}}^r-\underbrace{\hat{\utilde{\mathbf{m}}}^r\cdot\hat{\utilde{\mathbf{m}}}^r}_{\displaystyle1}+1\]
	\[\lambda^2=\hat{\utilde{\mathbf{m}}}^r\cdot\underline{\mathbf{C}}\hat{\utilde{\mathbf{m}}}^r\]
	
	Logo,
	\begin{equation}
		\lambda(\hat{\utilde{\mathbf{m}}}^r)=\sqrt{\hat{\utilde{\mathbf{m}}}^r\cdot\underline{\mathbf{C}}\hat{\utilde{\mathbf{m}}}^r}
	\end{equation}
	
	Para o alongamento linear,
	\[\varepsilon_l(\hat{\utilde{\mathbf{m}}}^r)=\lambda-1\]
	\begin{equation}\label{eq-alongamento-linear-1}
		\varepsilon_l(\hat{\utilde{\mathbf{m}}}^r)=\sqrt{\hat{\utilde{\mathbf{m}}}^r\cdot\underline{\mathbf{C}}\hat{\utilde{\mathbf{m}}}^r}-1
	\end{equation}
	
	Para a distorção, a partir da Equação \eqref{eq-distorcao-fibra-1},
	\[\sin\gamma=\frac{\hat{\utilde{\mathbf{b}}}^r\cdot\underline{\mathbf{F}}^{\text{T}}\underline{\mathbf{F}}\hat{\utilde{\mathbf{a}}}^r}{\sqrt{\hat{\utilde{\mathbf{a}}}^r\cdot\underline{\mathbf{F}}^{\text{T}}\underline{\mathbf{F}}\hat{\utilde{\mathbf{a}}}^r}\sqrt{\hat{\utilde{\mathbf{b}}}^r\cdot\underline{\mathbf{F}}^{\text{T}}\underline{\mathbf{F}}\hat{\utilde{\mathbf{b}}}^r}}\]
	
	Como $\underline{\mathbf{F}}^{\text{T}}\underline{\mathbf{F}}=\underline{\mathbf{C}}$,
	\[\sin\gamma=\frac{\hat{\utilde{\mathbf{b}}}^r\cdot\underline{\mathbf{C}}\hat{\utilde{\mathbf{a}}}^r}{\sqrt{\hat{\utilde{\mathbf{a}}}^r\cdot\underline{\mathbf{C}}\hat{\utilde{\mathbf{a}}}^r}\sqrt{\hat{\utilde{\mathbf{b}}}^r\cdot\underline{\mathbf{C}}\hat{\utilde{\mathbf{b}}}^r}}\]
	\[\sin\gamma=\frac{\hat{\utilde{\mathbf{b}}}^r\cdot\underline{\mathbf{C}}\hat{\utilde{\mathbf{a}}}^r}{\lambda(\hat{\utilde{\mathbf{a}}}^r)\lambda(\hat{\utilde{\mathbf{b}}}^r)}\]
	
	Ou seja, pode-se concluir que $\underline{\mathbf{C}}$ é simétrico.
	
	\subsubsection{Interpretação de $\underline{\mathbf{C}}$ na base ortonormal}
	\[\lambda(\utilde{\mathbf{\hat{e}_1}})=\sqrt{\utilde{\mathbf{\hat{e}_1}}\cdot\underline{\mathbf{C}}\utilde{\mathbf{\hat{e}_1}}}=\sqrt{C_{11}}\therefore\lambda^2(\utilde{\mathbf{\hat{e}_1}})=C_{11}\]
	\[\lambda(\utilde{\mathbf{\hat{e}_2}})=\sqrt{\utilde{\mathbf{\hat{e}_2}}\cdot\underline{\mathbf{C}}\utilde{\mathbf{\hat{e}_2}}}=\sqrt{C_{22}}\therefore\lambda^2(\utilde{\mathbf{\hat{e}_2}})=C_{22}\]
	\[\lambda(\utilde{\mathbf{\hat{e}_3}})=\sqrt{\utilde{\mathbf{\hat{e}_3}}\cdot\underline{\mathbf{C}}\utilde{\mathbf{\hat{e}_3}}}=\sqrt{C_{33}}\therefore\lambda^2(\utilde{\mathbf{\hat{e}_3}})=C_{33}\]
	
	Ou seja, a diagonal de $\underline{\mathbf{C}}$ guarda o alongamento quadrático.
	
	\[\sin(\utilde{\mathbf{\hat{e}_1}},\utilde{\mathbf{\hat{e}_2}})=\frac{\utilde{\mathbf{\hat{e}_1}}\cdot\underline{\mathbf{C}}\utilde{\mathbf{\hat{e}_2}}}{\lambda(\utilde{\mathbf{\hat{e}_1}})\lambda(\utilde{\mathbf{\hat{e}_2}})}=\frac{C_{12}}{\lambda(\utilde{\mathbf{\hat{e}_1}})\lambda(\utilde{\mathbf{\hat{e}_2}})}\]
	
	\subsection{Deslocamentos infinitesimais}
	
	Relembrando de $\underline{\mathbf{E}}$ na forma diferencial,
	\[\underline{\mathbf{E}}=\frac{1}{2}\left( \frac{\partial u^i}{\partial x^{rj}}+\frac{\partial u^j}{\partial x^{ri}}+\frac{\partial u^k}{\partial x^{ri}}\frac{\partial u^k}{\partial x^{rj}} \right)\]
	
	Para deslocamentos infinitesimais, pode-se desprezar o produto contido na expressão acima. Ficando como,
	\[E_{ij}^l=\frac{1}{2}\left( \frac{\partial u^i}{\partial x^{rj}}+\frac{\partial u^j}{\partial x^{ri}}\right)\]
	
	Ou seja, obtém-se o Tensor das Deformações infinitesimais linearizado, denotado por $\underline{\mathbf{E}}^l$ ou $E_{ij}^l$. Sendo,
	\[\underline{\mathbf{E}}^l=\frac{1}{2}(\nabla u^{\text{T}}+\nabla u)\]
	
	Agora, lembrando do alongamento linear em função do vetor unitário $\hat{\utilde{\mathbf{m}}}^r$ (direção de uma fibra na configuração de referência),
	\[\varepsilon_l(\hat{\utilde{\mathbf{m}}}^r)=\sqrt{1+2(\hat{\utilde{\mathbf{m}}}^r\cdot\underline{\mathbf{E}}\hat{\utilde{\mathbf{m}}}^r)}-1\]
	
	Elevando ambos os lados ao quadrado,
	\[(\varepsilon_l+1)^2=1+2(\hat{\utilde{\mathbf{m}}}^r\cdot\underline{\mathbf{E}}\hat{\utilde{\mathbf{m}}}^r)\]
	\[\varepsilon_l^2+2\varepsilon_l+1=1+2(\hat{\utilde{\mathbf{m}}}^r\cdot\underline{\mathbf{E}}\hat{\utilde{\mathbf{m}}}^r)\]
	
	Desprezando a ordem superior,
	\[\varepsilon_l=\hat{\utilde{\mathbf{m}}}^r\cdot\underline{\mathbf{E}}\hat{\utilde{\mathbf{m}}}^r\]
	
	Ou seja, $\underline{\mathbf{E}}$ tende para $\underline{\mathbf{E}}^l$ para deslocamentos lineares, portanto,
	\[\varepsilon_l(\hat{\utilde{\mathbf{m}}}^r)=\hat{\utilde{\mathbf{m}}}^r\cdot\underline{\mathbf{E}}^l\hat{\utilde{\mathbf{m}}}^r\]
	
	E para a distorção,
	\[\sin\gamma(\hat{\utilde{\mathbf{a}}}^r, \hat{\utilde{\mathbf{b}}}^r)=\frac{2(\hat{\utilde{\mathbf{b}}}^r\cdot\underline{\mathbf{E}}\hat{\utilde{\mathbf{a}}}^r)}{\sqrt{2(\hat{\utilde{\mathbf{a}}}^r\cdot\underline{\mathbf{E}}\hat{\utilde{\mathbf{a}}}^r)+1}\sqrt{2(\hat{\utilde{\mathbf{b}}}^r\cdot\underline{\mathbf{E}}\hat{\utilde{\mathbf{b}}}^r)+1}}\]
	
	Admitindo que o denominador acima será muito pequeno em relação a $1$,
	\[\sin\gamma(\hat{\utilde{\mathbf{a}}}^r, \hat{\utilde{\mathbf{b}}}^r)=2(\hat{\utilde{\mathbf{b}}}^r\cdot\underline{\mathbf{E}}^l\hat{\utilde{\mathbf{a}}}^r)\]
	
	\textbf{Exercício} - Capítulo 3/página 108 do livro \textit{The Mechanics of Solids and Structures - Hierarchical Modeling and The Finite Element Process of Solution (Bucalem, Bathe):}
	
	Configuração deformada para um bloco de aresta $a$,
	\[
	\begin{cases}
		x^1=x^{r1}+\tan\beta x^{r2} \\ x^2=x^{r2} \\ x^3=x^{r3}
	\end{cases}
	\]
	
	//Inserir desenho
	
	\begin{enumerate}
		\item Calcular o campo de deslocamentos e deformações normais ($\varepsilon_l$) nas direções ($\utilde{\mathbf{\hat{e}_1}}$), ($\utilde{\mathbf{\hat{e}_2}}$), ($\utilde{\mathbf{m_1}}=\frac{\sqrt{2}}{2}\utilde{\mathbf{\hat{e}_1}}+\frac{\sqrt{2}}{2}\utilde{\mathbf{\hat{e}_2}}$) e ($\utilde{\mathbf{m_2}}=-\frac{\sqrt{2}}{2}\utilde{\mathbf{\hat{e}_1}}+\frac{\sqrt{2}}{2}\utilde{\mathbf{\hat{e}_2}}$).
	\item Calcular as deformações por cisalhamento para os pares de fibras nas direções ($\utilde{\mathbf{\hat{e}_1}}$, $\utilde{\mathbf{\hat{e}_2}}$) e ($\utilde{\mathbf{m_1}}$, $\utilde{\mathbf{m_2}}$).
	\item Repetir os itens 1 e 2 assumindo um $\beta$ pequeno, obter os resultados usando a teoria dos pequenos deslocamentos, e mostrar que estes mesmos resultados são também obtidos de 1 e 2.
	\end{enumerate}
	
	\textbf{Solução}:
	
	\begin{enumerate}
		\item 

		O campo de deslocamentos é dado por,
		\[u^i=x^i-x^{ri}\]
	
		Aplicando para as coordenadas,
		\[u^1=x^1-x^{r1}=(x^{r1}+\tan\beta x^{r2})-x^{r1}=\tan\beta x^{r2}\]
		\[u^2=x^2-x^{r2}=x^{r2}-x^{r2}=0\]
		\[u^3=x^2-x^{r2}=x^{r3}-x^{r3}=0\]
	
		Considerando grandes deslocamentos, precisamos calcular o gradiente das deformações ($\underline{\mathbf{F}}$)
		\[
			\underline{\mathbf{F}}
			=
			\begin{bmatrix}
				1 & \tan\beta & 0 \\
				0 & 1 & 0 \\
				0 & 0 & 1
			\end{bmatrix}			
		\]
		Nota-se aqui que através do bloco, os elementos de F são constantes (independentes de $x^{r1}$, $x^{r2}$ e $x^{r3}$).
	
		Então,
		\[
			\underline{\mathbf{F}}^{\text{T}}\underline{\mathbf{F}}
			=
			\begin{bmatrix}
				1 & 0 & 0 \\
				\tan\beta & 1 & 0 \\
				0 & 0 & 1
			\end{bmatrix}
			\begin{bmatrix}
				1 & \tan\beta & 0 \\
				0 & 1 & 0 \\
				0 & 0 & 1
			\end{bmatrix}
			=
			\begin{bmatrix}
				1 & \tan\beta & 0 \\
				\tan\beta & \tan^2\beta + 1 & 0 \\
				0 & 0 & 1
			\end{bmatrix}
		\]
	
		Usando a Equação \eqref{eq-alongamento-linear-1} (alongamento linear) para as fibras do item 1:
	
		Para $\utilde{\mathbf{\hat{e}_1}}$,
		\[\varepsilon_l(\utilde{\mathbf{\hat{e}_1}})=\sqrt{\utilde{\mathbf{\hat{e}_1}}\cdot\underline{\mathbf{C}}\utilde{\mathbf{\hat{e}_1}}}-1\]
		\[
			\utilde{\mathbf{\hat{e}_1}}\cdot\underline{\mathbf{C}}\utilde{\mathbf{\hat{e}_1}}
			=
			\begin{Bmatrix}
				1 & 0 & 0
			\end{Bmatrix}
			\cdot
			\begin{bmatrix}
				1 & \tan\beta & 0 \\
				\tan\beta & \tan^2\beta + 1 & 0 \\
				0 & 0 & 1
			\end{bmatrix}
			\begin{Bmatrix}
				1 \\ 0 \\ 0
			\end{Bmatrix}
		\]
		\[
			\utilde{\mathbf{\hat{e}_1}}\cdot\underline{\mathbf{C}}\utilde{\mathbf{\hat{e}_1}}
			=
			\begin{Bmatrix}
				1 & 0 & 0
			\end{Bmatrix}
			\cdot
			\begin{Bmatrix}
				1 \\ \tan\beta \\ 0
			\end{Bmatrix}
			=
			1
		\]
		\[\varepsilon_l(\utilde{\mathbf{\hat{e}_1}})=\sqrt{1}-1=0\]
	
		Para $\utilde{\mathbf{\hat{e}_2}}$,
		\[\varepsilon_l(\utilde{\mathbf{\hat{e}_2}})=\sqrt{\utilde{\mathbf{\hat{e}_2}}\cdot\underline{\mathbf{C}}\utilde{\mathbf{\hat{e}_2}}}-1\]
		\[
			\utilde{\mathbf{\hat{e}_2}}\cdot\underline{\mathbf{C}}\utilde{\mathbf{\hat{e}_2}}
			=
			\begin{Bmatrix}
				0 & 1 & 0
			\end{Bmatrix}
			\cdot
			\begin{bmatrix}
				1 & \tan\beta & 0 \\
				\tan\beta & \tan^2\beta + 1 & 0 \\
				0 & 0 & 1
			\end{bmatrix}
			\begin{Bmatrix}
				0 \\ 1 \\ 0
			\end{Bmatrix}
		\]
		\[
			\utilde{\mathbf{\hat{e}_2}}\cdot\underline{\mathbf{C}}\utilde{\mathbf{\hat{e}_2}}
			=
			\begin{Bmatrix}
				0 & 1 & 0
			\end{Bmatrix}
			\cdot
			\begin{Bmatrix}
				\tan\beta \\ \tan^2\beta + 1 \\ 0
			\end{Bmatrix}
			=
			\tan^2\beta + 1
		\]
		\[\varepsilon_l(\utilde{\mathbf{\hat{e}_2}})=\sqrt{\tan^2\beta + 1}-1\]
	
		Para $\utilde{\mathbf{m_1}}$ (sendo $\utilde{\mathbf{m_1}}$ unitário, obrigatoriamente),
		\[\varepsilon_l(\utilde{\mathbf{m_1}})=\sqrt{\utilde{\mathbf{m_1}}\cdot\underline{\mathbf{C}}\utilde{\mathbf{m_1}}}-1\]
		\[
			\utilde{\mathbf{m_1}}\cdot\underline{\mathbf{C}}\utilde{\mathbf{m_1}}
			=
			\begin{Bmatrix}
				\frac{\sqrt{2}}{2} & \frac{\sqrt{2}}{2} & 0
			\end{Bmatrix}
			\cdot
			\begin{bmatrix}
				1 & \tan\beta & 0 \\
				\tan\beta & \tan^2\beta + 1 & 0 \\
				0 & 0 & 1
			\end{bmatrix}
			\begin{Bmatrix}
				\frac{\sqrt{2}}{2} \\ \frac{\sqrt{2}}{2} \\ 0
			\end{Bmatrix}
		\]
		\[
			\utilde{\mathbf{m_1}}\cdot\underline{\mathbf{C}}\utilde{\mathbf{m_1}}
			=
			\begin{Bmatrix}
				\frac{\sqrt{2}}{2} & \frac{\sqrt{2}}{2} & 0
			\end{Bmatrix}
			\cdot
			\begin{Bmatrix}
				\frac{\sqrt{2}}{2}(\tan\beta+1) \\ \frac{\sqrt{2}}{2}(\tan^2\beta+1+\tan\beta) \\ 0
			\end{Bmatrix}
		\]
		\[\utilde{\mathbf{m_1}}\cdot\underline{\mathbf{C}}\utilde{\mathbf{m_1}}=\frac{1}{2}(\tan\beta+1)+\frac{1}{2}(\tan^2\beta+1+\tan\beta)\]
		\[\utilde{\mathbf{m_1}}\cdot\underline{\mathbf{C}}\utilde{\mathbf{m_1}}=\tan\beta+1+\frac{\tan^2\beta}{2}\]
		\[\varepsilon_l(\utilde{\mathbf{m_1}})=\sqrt{\tan\beta+1+\frac{\tan^2\beta}{2}}-1\]
	
		Para $\utilde{\mathbf{m_2}}$ (sendo $\utilde{\mathbf{m_2}}$ unitário, obrigatoriamente),
		\[\varepsilon_l(\utilde{\mathbf{m_2}})=\sqrt{\utilde{\mathbf{m_2}}\cdot\underline{\mathbf{C}}\utilde{\mathbf{m_2}}}-1\]
		\[
			\utilde{\mathbf{m_2}}\cdot\underline{\mathbf{C}}\utilde{\mathbf{m_2}}
			=
			\begin{Bmatrix}
				-\frac{\sqrt{2}}{2} & \frac{\sqrt{2}}{2} & 0
			\end{Bmatrix}
			\cdot
			\begin{bmatrix}
				1 & \tan\beta & 0 \\
				\tan\beta & \tan^2\beta + 1 & 0 \\
				0 & 0 & 1
			\end{bmatrix}
			\begin{Bmatrix}
				-\frac{\sqrt{2}}{2} \\ \frac{\sqrt{2}}{2} \\ 0
			\end{Bmatrix}
		\]
		\[
			\utilde{\mathbf{m_2}}\cdot\underline{\mathbf{C}}\utilde{\mathbf{m_2}}
			=
			\begin{Bmatrix}
				-\frac{\sqrt{2}}{2} & \frac{\sqrt{2}}{2} & 0
			\end{Bmatrix}
			\cdot
			\begin{Bmatrix}
				\frac{\sqrt{2}}{2}(\tan\beta-1) \\ \frac{\sqrt{2}}{2}(\tan^2\beta+1-\tan\beta) \\ 0
			\end{Bmatrix}
		\]
		\[\utilde{\mathbf{m_2}}\cdot\underline{\mathbf{C}}\utilde{\mathbf{m_2}}=-\frac{1}{2}(\tan\beta-1)+\frac{1}{2}(\tan^2\beta+1-\tan\beta)\]
		\[\utilde{\mathbf{m_2}}\cdot\underline{\mathbf{C}}\utilde{\mathbf{m_2}}=-\tan\beta+1+\frac{\tan^2\beta}{2}\]
		\[\varepsilon_l(\utilde{\mathbf{m_2}})=\sqrt{1+\frac{\tan^2\beta}{2}-\tan\beta}-1\]
	
		\item As tensões de cisalhamento podem ser calculadas como segue,
		
		Sabendo que,
		\[\varepsilon_l(\hat{\utilde{\mathbf{m}}}^r)=\sqrt{\hat{\utilde{\mathbf{m}}}^r\cdot\underline{\mathbf{C}}\hat{\utilde{\mathbf{m}}}^r}-1\]
		\[1+\varepsilon_l(\hat{\utilde{\mathbf{m}}}^r)=\sqrt{\hat{\utilde{\mathbf{m}}}^r\cdot\underline{\mathbf{C}}\hat{\utilde{\mathbf{m}}}^r}\]
		
		Substituindo na Equação \eqref{eq-distorcao-fibra-1},
		\[
		\sin\gamma=\frac{\hat{\utilde{\mathbf{b}}}^r\cdot\underline{\mathbf{C}}\hat{\utilde{\mathbf{a}}}^r}{\sqrt{1+\varepsilon_l(\hat{\utilde{\mathbf{a}}}^r)}\sqrt{1+\varepsilon_l(\hat{\utilde{\mathbf{b}}}^r)}}		
		\]
		
		Para o par de fibras ($\utilde{\mathbf{\hat{e}_1}}$, $\utilde{\mathbf{\hat{e}_2}}$),
		
	\end{enumerate}
	
	\subsection{Movimentos rígidos: Rotações}
	
	//Inserir imagem
	
	Para a figura acima em uma base ortonormal e o eixo de rotação $\utilde{\mathbf{\hat{e}}}=\utilde{\mathbf{\hat{e}_3}}$, $\theta$ é a magnitude de rotação, $\utilde{\theta}=\theta\utilde{\mathbf{\hat{e}}}$ ($\utilde{\theta}$ define totalmente a rotação) e $||\utilde{\mathbf{\hat{e}}}||=1$.
	
	Lembrando do exercício de rotação da primeira aula,
	\[
		\underbrace{\begin{Bmatrix}
			x^1 \\ x^2 \\ x^3
		\end{Bmatrix}}_{\displaystyle\utilde{\mathbf{x}}}
		=
		\underbrace{
		\begin{bmatrix}
			\cos\varphi & -\sin\varphi & 0 \\
			\sin\varphi & \cos\varphi & 0 \\
			0 & 0 & 1
		\end{bmatrix}}_{\displaystyle\underline{\mathbf{Q}}}
		\underbrace{
		\begin{Bmatrix}
			x^{r1} \\ x^{r2} \\ x^{r3}
		\end{Bmatrix}}_{\displaystyle\utilde{\mathbf{x}}^r}		
	\]
	
	Onde $\underline{\mathbf{Q}}$ é o operador (tensor) ortogonal.
	
	Pode-se definir um operador ortogonal de forma equivalente por uma das relações abaixo,
	\begin{itemize}
		\item $||\underline{\mathbf{Q}}\utilde{\mathbf{w}}||=||\utilde{\mathbf{w}}||\;\forall\;\utilde{\mathbf{w}}$ (Não muda o comprimento).
		\item $\underline{\mathbf{Q}}\underline{\mathbf{Q}}^{\text{T}}=\underline{\mathbf{Q}}^{\text{T}}\underline{\mathbf{Q}}=\underline{\mathbf{I}}\implies\underline{\mathbf{Q}}^{-1}=\underline{\mathbf{Q}}^{\text{T}}$
		\item A partir de item acima,
		\[\det(\underline{\mathbf{Q}}\underline{\mathbf{Q}}^{\text{T}})=\det(\underline{\mathbf{I}})\]
		\[\det(\underline{\mathbf{Q}})\det(\underline{\mathbf{Q}}^{\text{T}})=1\]
		\[(\det\underline{\mathbf{Q}})^2=1\]
		\[\det(\underline{\mathbf{Q}})=\pm1\]
	\end{itemize}
	
	Pode-se demonstrar que para qualquer tensor ortogonal com determinante positivo ($\det(\underline{\mathbf{Q}})=1$), a rotação é descrita através da equação,
	\[\utilde{\mathbf{x}}=\underline{\mathbf{Q}}\utilde{\mathbf{x}}^r\]
	
	\textbf{Exercício} - Capítulo 3/página 144 do livro \textit{The Mechanics of Solids and Structures - Hierarchical Modeling and The Finite Element Process of Solution (Bucalem, Bathe):}
	
	Considere o tensor $\underline{\mathbf{Q}}$ na base ortogonal,
	\[
		\underline{\mathbf{Q}}
		=
		\begin{bmatrix}
			\frac{4}{9}+\frac{5\sqrt{3}}{18} & \frac{5}{18}+\frac{2\sqrt{3}}{9} & \frac{5}{9}-\frac{\sqrt{3}}{9} \\
			\frac{11}{18}-\frac{2\sqrt{3}}{9} & \frac{4}{9}+\frac{5\sqrt{3}}{18} & -\frac{1}{9}-\frac{\sqrt{3}}{9} \\
			-\frac{1}{9}-\frac{\sqrt{3}}{9} & \frac{5}{9}-\frac{\sqrt{3}}{9} & \frac{1}{9}+\frac{4\sqrt{3}}{9}
		\end{bmatrix}
	\]
	
	\begin{enumerate}
		\item Verificar se $\underline{\mathbf{Q}}$ é um tensor ortogonal;
		\item Obter o eixo e magnitude dada por $\underline{\mathbf{Q}}$.
	\end{enumerate}
	
	Solução:
	
	\begin{enumerate}
		\item Deve-se verificar se $\underline{\mathbf{Q}}\underline{\mathbf{Q}}^{\text{T}}=1$
		\item Considerando que $\utilde{\mathbf{\hat{e}}}$ é um vetor unitário ($||\utilde{\mathbf{\hat{e}}}||=1$) na direção do eixo de rotação, então:
		\[\underline{\mathbf{Q}}\utilde{\mathbf{\hat{e}}}=\utilde{\mathbf{\hat{e}}}\;\text{ (Não muda a magnitude)}\]
		\[\underline{\mathbf{Q}}\utilde{\mathbf{\hat{e}}}-\utilde{\mathbf{\hat{e}}}=0\]
		\[(\underline{\mathbf{Q}}-\underline{\mathbf{I}})\utilde{\mathbf{\hat{e}}}=0\]
		
		Na forma matricial,
		\[
			\begin{bmatrix}
				Q_{11}-1 & Q_{12} & Q_{13} \\
				Q_{21} & Q_{22}-1 & Q_{23} \\
				Q_{31} & Q_{32} & Q_{33}-1
			\end{bmatrix}
			\begin{Bmatrix}
				e_1 \\ e_2 \\ e_3
			\end{Bmatrix}
			=
			\begin{Bmatrix}
				0 \\ 0 \\ 0
			\end{Bmatrix}
		\]
		
		Resolvendo o sistema
		\[
			\utilde{\mathbf{\hat{e}}}
			=
			\begin{Bmatrix}
				2/3 \\ 2/3 \\ 1/3
			\end{Bmatrix}
		\]
		
		Verificando se não houve falha no processo de cálculo,
		\[||\utilde{\mathbf{\hat{e}}}||=\sqrt{(2/3)^2+(2/3)^2+(1/3)^2}=1\]
		
		Adotando $\utilde{\mathbf{g}}$ como um vetor unitário ortogonal ao eixo de rotação e escolhendo qualquer um dos vetores da base,
		\[
			\utilde{\mathbf{g}}
			=
			\frac{\utilde{\mathbf{\hat{e}}}\times\utilde{\mathbf{\hat{e}_2}}}{||\utilde{\mathbf{\hat{e}}}\times\utilde{\mathbf{\hat{e}_2}}||}
			=
			\begin{Bmatrix}
				-\sqrt{5}/5 \\ 0 \\ \frac{2\sqrt{5}}{5}
			\end{Bmatrix}
		\]
		
		Aplicando a rotação em $\utilde{\mathbf{g}}$ para obter um vetor $\utilde{\mathbf{f}}$,
		\[
			\utilde{\mathbf{f}}
			=
			\underline{\mathbf{Q}}\utilde{\mathbf{g}}
			=
			\begin{Bmatrix}
				-\frac{\sqrt{15}}{10}+\frac{2\sqrt{5}}{15} \\ -\frac{5}{6} \\ \frac{\sqrt{15}}{5}+\frac{\sqrt{5}}{15}
			\end{Bmatrix}
		\]
		
		Encontrando o ângulo de rotação pelo produto escalar entre $\utilde{\mathbf{g}}$ e $\utilde{\mathbf{f}}$,
		\[\utilde{\mathbf{g}}\cdot\utilde{\mathbf{f}}=||\utilde{\mathbf{g}}||\;||\utilde{\mathbf{f}}||\cos\theta\]
		\[\cos\theta=\frac{\sqrt{3}}{2}\therefore\theta=\text{30\textdegree}\]
		
	\end{enumerate}
	