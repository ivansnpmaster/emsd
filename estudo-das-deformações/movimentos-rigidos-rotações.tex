//Inserir imagem

Para a figura acima em uma base ortonormal e o eixo de rotação $\utilde{\mathbf{\hat{e}}}=\utilde{\mathbf{\hat{e}_3}}$, $\theta$ é a magnitude de rotação, $\utilde{\theta}=\theta\utilde{\mathbf{\hat{e}}}$ ($\utilde{\theta}$ define totalmente a rotação) e $||\utilde{\mathbf{\hat{e}}}||=1$.

Lembrando do exercício de rotação da primeira aula,
\[
    \underbrace{\begin{Bmatrix}
        x^1 \\ x^2 \\ x^3
    \end{Bmatrix}}_{\displaystyle\utilde{\mathbf{x}}}
    =
    \underbrace{
    \begin{bmatrix}
        \cos\varphi & -\sin\varphi & 0 \\
        \sin\varphi & \cos\varphi & 0 \\
        0 & 0 & 1
    \end{bmatrix}}_{\displaystyle\underline{\mathbf{Q}}}
    \underbrace{
    \begin{Bmatrix}
        x^{r1} \\ x^{r2} \\ x^{r3}
    \end{Bmatrix}}_{\displaystyle\utilde{\mathbf{x}}^r}		
\]

Onde $\underline{\mathbf{Q}}$ é o operador (tensor) ortogonal.

Pode-se definir um operador ortogonal de forma equivalente por uma das relações abaixo,
\begin{itemize}
    \item $||\underline{\mathbf{Q}}\utilde{\mathbf{w}}||=||\utilde{\mathbf{w}}||\;\forall\;\utilde{\mathbf{w}}$ (Não muda o comprimento).
    \item $\underline{\mathbf{Q}}\underline{\mathbf{Q}}^{\text{T}}=\underline{\mathbf{Q}}^{\text{T}}\underline{\mathbf{Q}}=\underline{\mathbf{I}}\implies\underline{\mathbf{Q}}^{-1}=\underline{\mathbf{Q}}^{\text{T}}$
    \item A partir de item acima,
    	\[\det(\underline{\mathbf{Q}}\underline{\mathbf{Q}}^{\text{T}})=\det(\underline{\mathbf{I}})\]
    	\[\det(\underline{\mathbf{Q}})\det(\underline{\mathbf{Q}}^{\text{T}})=1\]
    	\[(\det\underline{\mathbf{Q}})^2=1\]
    	\[\det(\underline{\mathbf{Q}})=\pm1\]
\end{itemize}

Pode-se demonstrar que para qualquer tensor ortogonal com determinante positivo ($\det(\underline{\mathbf{Q}})=1$), a rotação é descrita através da equação,
\[\utilde{\mathbf{x}}=\underline{\mathbf{Q}}\utilde{\mathbf{x}}^r\]

\textbf{Exercício} - Capítulo 3/página 144 do livro \textit{The Mechanics of Solids and Structures - Hierarchical Modeling and The Finite Element Process of Solution (Bucalem, Bathe):}

Considere o tensor $\underline{\mathbf{Q}}$ na base ortogonal,
\[
    \underline{\mathbf{Q}}
    =
    \begin{bmatrix}
        \frac{4}{9}+\frac{5\sqrt{3}}{18} & \frac{5}{18}+\frac{2\sqrt{3}}{9} & \frac{5}{9}-\frac{\sqrt{3}}{9} \\
        \frac{11}{18}-\frac{2\sqrt{3}}{9} & \frac{4}{9}+\frac{5\sqrt{3}}{18} & -\frac{1}{9}-\frac{\sqrt{3}}{9} \\
        -\frac{1}{9}-\frac{\sqrt{3}}{9} & \frac{5}{9}-\frac{\sqrt{3}}{9} & \frac{1}{9}+\frac{4\sqrt{3}}{9}
    \end{bmatrix}
\]

\begin{enumerate}
    \item Verificar se $\underline{\mathbf{Q}}$ é um tensor ortogonal;
    \item Obter o eixo e magnitude dada por $\underline{\mathbf{Q}}$.
\end{enumerate}

Solução:

\begin{enumerate}
    \item Deve-se verificar se $\underline{\mathbf{Q}}\underline{\mathbf{Q}}^{\text{T}}=1$
    \item Considerando que $\utilde{\mathbf{\hat{e}}}$ é um vetor unitário ($||\utilde{\mathbf{\hat{e}}}||=1$) na direção do eixo de rotação, então:
    	\[\underline{\mathbf{Q}}\utilde{\mathbf{\hat{e}}}=\utilde{\mathbf{\hat{e}}}\;\text{ (Não muda a magnitude)}\]
    	\[\underline{\mathbf{Q}}\utilde{\mathbf{\hat{e}}}-\utilde{\mathbf{\hat{e}}}=0\]
    	\[(\underline{\mathbf{Q}}-\underline{\mathbf{I}})\utilde{\mathbf{\hat{e}}}=0\]
    
    	Na forma matricial,
    	\[
     	   \begin{bmatrix}
    	        Q_{11}-1 & Q_{12} & Q_{13} \\
      	      Q_{21} & Q_{22}-1 & Q_{23} \\
     	       Q_{31} & Q_{32} & Q_{33}-1
     	   \end{bmatrix}
     	   \begin{Bmatrix}
     	       e_1 \\ e_2 \\ e_3
     	   \end{Bmatrix}
     	   =
     	   \begin{Bmatrix}
     	       0 \\ 0 \\ 0
     	   \end{Bmatrix}
    	\]
    
    	Resolvendo o sistema
    	\[
    	    \utilde{\mathbf{\hat{e}}}
    	    =
    	    \begin{Bmatrix}
    	        2/3 \\ 2/3 \\ 1/3
    	    \end{Bmatrix}
    	\]
    
    	Verificando se não houve falha no processo de cálculo,
    	\[||\utilde{\mathbf{\hat{e}}}||=\sqrt{(2/3)^2+(2/3)^2+(1/3)^2}=1\]
    
    	Adotando $\utilde{\mathbf{g}}$ como um vetor unitário ortogonal ao eixo de rotação e escolhendo qualquer um dos vetores da base,
    	\[
        	\utilde{\mathbf{g}}
        	=
        	\frac{\utilde{\mathbf{\hat{e}}}\times\utilde{\mathbf{\hat{e}_2}}}{||\utilde{\mathbf{\hat{e}}}\times\utilde{\mathbf{\hat{e}_2}}||}
        	=
        	\begin{Bmatrix}
        	    -\sqrt{5}/5 \\ 0 \\ \frac{2\sqrt{5}}{5}
        	\end{Bmatrix}
    	\]
    
    	Aplicando a rotação em $\utilde{\mathbf{g}}$ para obter um vetor $\utilde{\mathbf{f}}$,
    	\[
        	\utilde{\mathbf{f}}
        	=
        	\underline{\mathbf{Q}}\utilde{\mathbf{g}}
        	=
        	\begin{Bmatrix}
        	    -\frac{\sqrt{15}}{10}+\frac{2\sqrt{5}}{15} \\ -\frac{5}{6} \\ \frac{\sqrt{15}}{5}+\frac{\sqrt{5}}{15}
        	\end{Bmatrix}
    	\]
    
    	Encontrando o ângulo de rotação pelo produto escalar entre $\utilde{\mathbf{g}}$ e $\utilde{\mathbf{f}}$,
    	\[\utilde{\mathbf{g}}\cdot\utilde{\mathbf{f}}=||\utilde{\mathbf{g}}||\;||\utilde{\mathbf{f}}||\cos\theta\]
    	\[\cos\theta=\frac{\sqrt{3}}{2}\therefore\theta=\text{30\textdegree}\]
    
\end{enumerate}