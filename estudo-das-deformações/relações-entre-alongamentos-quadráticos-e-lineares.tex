O alongamento quadrático,
\[\lambda=\sqrt{2\varepsilon_q+1}\]

Elevando ambos os lados ao quadrado,
\[\lambda^2=2\varepsilon_q+1=2(\hat{\utilde{\mathbf{m}}}^r\cdot\underline{\mathbf{E}}\hat{\utilde{\mathbf{m}}}^r)+1\]

Como $\underline{\mathbf{C}}=\underline{\mathbf{F}}^{\text{T}}\underline{\mathbf{F}}$, $\underline{\mathbf{E}}$ pode ser reescrito como,
\begin{equation}\label{eq-green-lagrange-em-funcao-de-cauchy-green}
    \underline{\mathbf{E}}=\frac{1}{2}(\underline{\mathbf{C}}-\underline{\mathbf{I}})
\end{equation}

Pode-se reescrever $\lambda^2$ como,
\[\lambda^2=2\left\{\hat{\utilde{\mathbf{m}}}^r\cdot\left[\frac{1}{2}(\underline{\mathbf{C}}-\underline{\mathbf{I}})\right]\hat{\utilde{\mathbf{m}}}^r\right\}+1\]
\[\lambda^2=\hat{\utilde{\mathbf{m}}}^r\cdot\underline{\mathbf{C}}\hat{\utilde{\mathbf{m}}}^r-\hat{\utilde{\mathbf{m}}}^r\cdot\underline{\mathbf{I}}\hat{\utilde{\mathbf{m}}}^r+1\]
\[\lambda^2=\hat{\utilde{\mathbf{m}}}^r\cdot\underline{\mathbf{C}}\hat{\utilde{\mathbf{m}}}^r-\underbrace{\hat{\utilde{\mathbf{m}}}^r\cdot\hat{\utilde{\mathbf{m}}}^r}_{\displaystyle1}+1\]
\[\lambda^2=\hat{\utilde{\mathbf{m}}}^r\cdot\underline{\mathbf{C}}\hat{\utilde{\mathbf{m}}}^r\]

Logo,
\begin{equation}
    \lambda(\hat{\utilde{\mathbf{m}}}^r)=\sqrt{\hat{\utilde{\mathbf{m}}}^r\cdot\underline{\mathbf{C}}\hat{\utilde{\mathbf{m}}}^r}
\end{equation}

Para o alongamento linear,
\[\varepsilon_l(\hat{\utilde{\mathbf{m}}}^r)=\lambda-1\]
\begin{equation}\label{eq-alongamento-linear-1}
    \varepsilon_l(\hat{\utilde{\mathbf{m}}}^r)=\sqrt{\hat{\utilde{\mathbf{m}}}^r\cdot\underline{\mathbf{C}}\hat{\utilde{\mathbf{m}}}^r}-1
\end{equation}

Para a distorção, a partir da Equação \eqref{eq-distorcao-fibra-1},
\[\sin\gamma=\frac{\hat{\utilde{\mathbf{b}}}^r\cdot\underline{\mathbf{F}}^{\text{T}}\underline{\mathbf{F}}\hat{\utilde{\mathbf{a}}}^r}{\sqrt{\hat{\utilde{\mathbf{a}}}^r\cdot\underline{\mathbf{F}}^{\text{T}}\underline{\mathbf{F}}\hat{\utilde{\mathbf{a}}}^r}\sqrt{\hat{\utilde{\mathbf{b}}}^r\cdot\underline{\mathbf{F}}^{\text{T}}\underline{\mathbf{F}}\hat{\utilde{\mathbf{b}}}^r}}\]

Como $\underline{\mathbf{F}}^{\text{T}}\underline{\mathbf{F}}=\underline{\mathbf{C}}$,
\[\sin\gamma=\frac{\hat{\utilde{\mathbf{b}}}^r\cdot\underline{\mathbf{C}}\hat{\utilde{\mathbf{a}}}^r}{\displaystyle\sqrt{\hat{\utilde{\mathbf{a}}}^r\cdot\underline{\mathbf{C}}\hat{\utilde{\mathbf{a}}}^r}\sqrt{\hat{\utilde{\mathbf{b}}}^r\cdot\underline{\mathbf{C}}\hat{\utilde{\mathbf{b}}}^r}}\]
\[\sin\gamma=\frac{\hat{\utilde{\mathbf{b}}}^r\cdot\underline{\mathbf{C}}\hat{\utilde{\mathbf{a}}}^r}{\lambda(\hat{\utilde{\mathbf{a}}}^r)\lambda(\hat{\utilde{\mathbf{b}}}^r)}\]

Ou seja, pode-se concluir que $\underline{\mathbf{C}}$ é simétrico.