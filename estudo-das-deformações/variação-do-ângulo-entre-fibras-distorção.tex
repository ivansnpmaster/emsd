//Inserir imagem

\begin{itemize}
    \item $d\utilde{\mathbf{a}}^r$ e $d\utilde{\mathbf{b}}^r$ são inicialmente ortogonais;
    \item $\underline{\mathbf{F}}d\utilde{\mathbf{a}}^r$ tem a mesma direção de $d\utilde{\mathbf{a}}$ (hipótese);
    \item $\hat{\utilde{\mathbf{a}}}$ e $\hat{\utilde{\mathbf{b}}}$ são versores.
\end{itemize}

Lembrando que,
\[d\utilde{\mathbf{a}}=\underline{\mathbf{F}}d\utilde{\mathbf{a}}^r\text{ e }d\utilde{\mathbf{b}}=\underline{\mathbf{F}}d\utilde{\mathbf{b}}^r\]

A fim de encontrar novas relações, podemos definir o produto escalar,
\[d\utilde{\mathbf{a}}\cdot d\utilde{\mathbf{b}}=\underline{\mathbf{F}}d\utilde{\mathbf{a}}^r\cdot\underline{\mathbf{F}}d\utilde{\mathbf{b}}^r\]
\[d\utilde{\mathbf{a}}\cdot d\utilde{\mathbf{b}}=||\underline{\mathbf{F}}d\utilde{\mathbf{a}}^r||\;|| \underline{\mathbf{F}}d\utilde{\mathbf{b}}^r ||\cos\theta\]

Logo,
\begin{equation}\label{eq-variacao-angulo-fibras}
    \cos\theta=\sin\gamma=\frac{\underline{\mathbf{F}}d\utilde{\mathbf{a}}^r\cdot\underline{\mathbf{F}}d\utilde{\mathbf{b}}^r}{||\underline{\mathbf{F}}d\utilde{\mathbf{a}}^r||\;|| \underline{\mathbf{F}}d\utilde{\mathbf{b}}^r ||}
\end{equation}

E,
\[d\utilde{\mathbf{a}}^r=d\mathbf{a}^r\hat{\utilde{\mathbf{a}}}^r\]
\[d\utilde{\mathbf{b}}^r=d\mathbf{b}^r\hat{\utilde{\mathbf{b}}}^r\]
\[||\underline{\mathbf{F}}d\utilde{\mathbf{a}}^r||=\sqrt{\underline{\mathbf{F}}d\utilde{\mathbf{a}}^r\cdot\underline{\mathbf{F}}d\utilde{\mathbf{a}}^r}\]
\[||\underline{\mathbf{F}}d\utilde{\mathbf{b}}^r||=\sqrt{\underline{\mathbf{F}}d\utilde{\mathbf{b}}^r\cdot\underline{\mathbf{F}}d\utilde{\mathbf{b}}^r}\]

Substituindo as 4 relações acima na Equação \eqref{eq-variacao-angulo-fibras},
\[\sin\gamma=\frac{d\utilde{\mathbf{b}}^r\cdot\underline{\mathbf{F}}^{\text{T}}\underline{\mathbf{F}}d\utilde{\mathbf{a}}^r}{\sqrt{d\utilde{\mathbf{a}}^r\cdot\underline{\mathbf{F}}^{\text{T}}\underline{\mathbf{F}}d\utilde{\mathbf{a}}^r}\sqrt{d\utilde{\mathbf{b}}^r\cdot\underline{\mathbf{F}}^{\text{T}}\underline{\mathbf{F}}d\utilde{\mathbf{b}}^r}}\]

\[\sin\gamma=\frac{(d\mathbf{b}^r\hat{\utilde{\mathbf{b}}}^r)\cdot[\underline{\mathbf{F}}^{\text{T}}\underline{\mathbf{F}}(d\mathbf{a}^r\hat{\utilde{\mathbf{a}}}^r)]}{\sqrt{(d\mathbf{a}^r\hat{\utilde{\mathbf{a}}}^r)\cdot[\underline{\mathbf{F}}^{\text{T}}\underline{\mathbf{F}}(d\mathbf{a}^r\hat{\utilde{\mathbf{a}}}^r)]}\sqrt{(d\mathbf{b}^r\hat{\utilde{\mathbf{b}}}^r)\cdot[\underline{\mathbf{F}}^{\text{T}}\underline{\mathbf{F}}(d\mathbf{b}^r\hat{\utilde{\mathbf{b}}}^r)]}}\]

\[\sin\gamma=\frac{(d\mathbf{b}^rd\mathbf{a}^r)[\hat{\utilde{\mathbf{b}}}^r\cdot\underline{\mathbf{F}}^{\text{T}}\underline{\mathbf{F}}\hat{\utilde{\mathbf{a}}}^r]}{(d\mathbf{b}^rd\mathbf{a}^r)\sqrt{\hat{\utilde{\mathbf{a}}}^r\cdot\underline{\mathbf{F}}^{\text{T}}\underline{\mathbf{F}}\hat{\utilde{\mathbf{a}}}^r}\sqrt{\hat{\utilde{\mathbf{b}}}^r\cdot\underline{\mathbf{F}}^{\text{T}}\underline{\mathbf{F}}\hat{\utilde{\mathbf{b}}}^r}}\]

\begin{equation}\label{eq-distorcao-fibra-1}
    \sin\gamma=\frac{\hat{\utilde{\mathbf{b}}}^r\cdot\underline{\mathbf{F}}^{\text{T}}\underline{\mathbf{F}}\hat{\utilde{\mathbf{a}}}^r}{\sqrt{\hat{\utilde{\mathbf{a}}}^r\cdot\underline{\mathbf{F}}^{\text{T}}\underline{\mathbf{F}}\hat{\utilde{\mathbf{a}}}^r}\sqrt{\hat{\utilde{\mathbf{b}}}^r\cdot\underline{\mathbf{F}}^{\text{T}}\underline{\mathbf{F}}\hat{\utilde{\mathbf{b}}}^r}}
\end{equation}		

Lembrando que,
\[\underline{\mathbf{E}}=\frac{1}{2}(\underline{\mathbf{F}}^{\text{T}}\underline{\mathbf{F}}-\underline{\mathbf{I}})\]
\[\underline{\mathbf{F}}^{\text{T}}\underline{\mathbf{F}}=2\underline{\mathbf{E}}+\underline{\mathbf{I}}\]

Substituindo a expressão acima na Equação \eqref{eq-distorcao-fibra-1},
\[\sin\gamma=\frac{\hat{\utilde{\mathbf{b}}}^r\cdot(2\underline{\mathbf{E}}+\underline{\mathbf{I}})\hat{\utilde{\mathbf{a}}}^r}{\sqrt{\hat{\utilde{\mathbf{a}}}^r\cdot(2\underline{\mathbf{E}}+\underline{\mathbf{I}})\hat{\utilde{\mathbf{a}}}^r}\sqrt{\hat{\utilde{\mathbf{b}}}^r\cdot(2\underline{\mathbf{E}}+\underline{\mathbf{I}})\hat{\utilde{\mathbf{b}}}^r}}\]
\[(2\underline{\mathbf{E}}+\underline{\mathbf{I}})\hat{\utilde{\mathbf{a}}}^r=2\underline{\mathbf{E}}\hat{\utilde{\mathbf{a}}}^r+\underline{\mathbf{I}}\hat{\utilde{\mathbf{a}}}^r=2\underline{\mathbf{E}}\hat{\utilde{\mathbf{a}}}^r+\hat{\utilde{\mathbf{a}}}^r\]

\[\sin\gamma=\frac{\hat{\utilde{\mathbf{b}}}^r\cdot(2\underline{\mathbf{E}}\hat{\utilde{\mathbf{a}}}^r+\hat{\utilde{\mathbf{a}}}^r)}{\sqrt{\hat{\utilde{\mathbf{a}}}^r\cdot(2\underline{\mathbf{E}}\hat{\utilde{\mathbf{a}}}^r+\hat{\utilde{\mathbf{a}}}^r)}\sqrt{\hat{\utilde{\mathbf{b}}}^r\cdot(2\underline{\mathbf{E}}\hat{\utilde{\mathbf{b}}}^r+\hat{\utilde{\mathbf{b}}}^r)}}\]

\[\sin\gamma=\frac{\hat{\utilde{\mathbf{b}}}^r\cdot2\underline{\mathbf{E}}\hat{\utilde{\mathbf{a}}}^r+\hat{\utilde{\mathbf{b}}}^r\cdot\hat{\utilde{\mathbf{a}}}^r}{\displaystyle\sqrt{\hat{\utilde{\mathbf{a}}}^r\cdot2\underline{\mathbf{E}}\hat{\utilde{\mathbf{a}}}^r+\hat{\utilde{\mathbf{a}}}^r\cdot\hat{\utilde{\mathbf{a}}}^r}\sqrt{\hat{\utilde{\mathbf{b}}}^r\cdot2\underline{\mathbf{E}}\hat{\utilde{\mathbf{b}}}^r+\hat{\utilde{\mathbf{b}}}^r\cdot\hat{\utilde{\mathbf{b}}}^r}}\]

Por fim,
\begin{equation}\label{eq-distorcao-fibra-2}
    \sin\gamma(\hat{\utilde{\mathbf{a}}}^r,\hat{\utilde{\mathbf{b}}}^r)=\frac{2(\hat{\utilde{\mathbf{b}}}^r\cdot\underline{\mathbf{E}}\hat{\utilde{\mathbf{a}}}^r)}{\displaystyle\sqrt{2(\hat{\utilde{\mathbf{a}}}^r\cdot\underline{\mathbf{E}}\hat{\utilde{\mathbf{a}}}^r)+1}\sqrt{2(\hat{\utilde{\mathbf{b}}}^r\cdot\underline{\mathbf{E}}\hat{\utilde{\mathbf{b}}}^r)+1}}
\end{equation}

Onde $\gamma$ é a distorção, \textit{i.e.}, a variação de ângulo entre duas fibras inicialmente ortogonais.