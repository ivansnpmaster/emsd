//Inserir imagem

O assunto agora são fibras; como as fibras sofrem deformação. Seja $d\vt{x}^r$ o vetor infinitesimal que representa uma fibra a partir do ponto $\vt{x}^r$; $d\vt{x}$ o vetor infinitesimal da mesma fibra agora na configuração deformada, partindo do ponto $\vt{x}$.
Para que o vetor $d\vt{x}^r$ deforme e se transforme em $d\vt{x}$, deve ocorrer uma transformação que depende de $\vt{x}^r+d\vt{x}^r$, \textit{i.e.}, uma transformação $\vt{u}(\vt{x}^r+d\vt{x}^r)$.

Algumas relações podem ser estabelecidas a partir da imagem acima, sendo:
\[\vt{u}(\vt{x}^r)=d\vt{x}^r+\vt{u}(\vt{x}^r+d\vt{x}^r)\]

E a fibra na configuração deformada:	
\[d\vt{x}=d\vt{x}^r+\vt{u}(\vt{x}^r+d\vt{x}^r)-\vt{u}(\vt{x}^r)\]

A equação acima na forma de componentes:
\[dx^i=dx^{ri}+u^i(x^{r1}+dx^{r1},\;x^{r2}+dx^{r2},\;x^{r3}+dx^{r3})-u^i(x^{r1},\;x^{r2},\;x^{r3})\]

Lembrando de cálculo com múltiplas variáveis:
\[u^i(x^{r1}+dx^{r1},\;x^{r2}+dx^{r2},\;x^{r3}+dx^{r3})-u^i(x^{r1},\;x^{r2},\;x^{r3})=\frac{\partial u^i}{\partial x^{r1}}dx^{r1}+\frac{\partial u^i}{\partial x^{r2}}dx^{r2}+\frac{\partial u^i}{\partial x^{r3}}dx^{r3}\]

Para $i=1,\;2$ e $3$. Essa iteração diz que cada coordenada $i$ depende de acontecimentos nas $3$ dimensões do espaço euclidiano.

Reescrevendo na forma matricial:
\[
	\underbrace{
	\begin{Bmatrix}
		dx^1 \\ dx^2 \\ dx^3
	\end{Bmatrix}
	}_{\displaystyle d\vt{x}}
	=
	\underbrace{
	\begin{Bmatrix}
		dx^{r1} \\ dx^{r2} \\ dx^{r3}
	\end{Bmatrix}
	}_{\displaystyle d\vt{x}^r}
	+
	\underbrace{
	\begin{bmatrix}
		\frac{\partial u^1}{\partial x^{r1}} & \frac{\partial u^1}{\partial x^{r2}} & \frac{\partial u^1}{\partial x^{r3}} \\
		\frac{\partial u^2}{\partial x^{r1}} & \frac{\partial u^2}{\partial x^{r2}} & \frac{\partial u^2}{\partial x^{r3}} \\
		\frac{\partial u^3}{\partial x^{r1}} & \frac{\partial u^3}{\partial x^{r2}} & \frac{\partial u^3}{\partial x^{r3}}
	\end{bmatrix}
	}_{\displaystyle\nabla\vt{u}=\ts{L}}
	\underbrace{
	\begin{Bmatrix}
		dx^{r1} \\ dx^{r2} \\ dx^{r3}
	\end{Bmatrix}
	}_{\displaystyle d\vt{x}^r}
\]

Onde $\nabla\vt{u}$ é o \textbf{gradiente dos deslocamentos}.

Logo,
\[d\vt{x}=d\vt{x}^r+\nabla\vt{u}\;d\vt{x}^r\]

Colocando $d\vt{x}^r$ em evidência, temos:
\[d\vt{x}=\underbrace{(\ts{I}+\nabla\vt{u})}_{\displaystyle \ts{F}}d\vt{x}^r\]

Onde $\ts{F}$ é o \textbf{gradiente das deformações}.

Portanto, temos:
\begin{equation}
	d\vt{x}=\ts{F}\;d\vt{x}^r
\end{equation}

Ou seja, a fibra deformada agora pode ser obtida a partir do gradiente dos deslocamentos ($\nabla\vt{u}$) e também a partir do gradiente das deformações ($\ts{F}$).

O gradiente das deformações pode ser reescrito como:
\[
	\ts{F}=
	\begin{bmatrix}
		1+\frac{\partial u^1}{\partial x^{r1}} & \frac{\partial u^1}{\partial x^{r2}} & \frac{\partial u^1}{\partial x^{r3}} \\
		\frac{\partial u^2}{\partial x^{r1}} & 1+\frac{\partial u^2}{\partial x^{r2}} & \frac{\partial u^2}{\partial x^{r3}} \\
		\frac{\partial u^3}{\partial x^{r1}} & \frac{\partial u^3}{\partial x^{r2}} & 1+\frac{\partial u^3}{\partial x^{r3}}
	\end{bmatrix}
\]

Ou ainda, usando notação indicial de Einstein:
\[F_{ij}=\nabla u_{ij}+\delta_{ij}\]

Onde $\delta_{ij}$ é o delta de Kronecker, \textit{i.e.}: $\delta_{ij}=\begin{cases} 1, & \text{se } i=j \\ 0, & \text{se } i\neq j \end{cases}$

E em notação indicial, temos:
\[\nabla u_{ij}=\frac{\partial u^i}{\partial x^{rj}}\]

Agora, expressemos:
\[\vt{u}=\vt{x}-\vt{x}^r\implies \vt{x}=\vt{u}+\vt{x}^r\]
\[x^i=u^i+x^{ri}\]

Derivando ambos os lados da notação indicial acima em relação a $\vt{x}^r$, temos:
\[\underbrace{\frac{\partial x^i}{\partial x^{rj}}}_{\displaystyle\ts{F}}=\underbrace{\frac{\partial u^i}{\partial x^{rj}}}_{\displaystyle\nabla u_{ij}}+\underbrace{\frac{\partial x^{ri}}{\partial x^{rj}}}_{\displaystyle\delta_{ij}}\]

Ou seja, agora o gradiente das deformações está em função da transformação e sua forma matricial é:
\[
	\ts{F}=
	\begin{bmatrix}
		\frac{\partial x^1}{\partial x^{r1}} & \frac{\partial x^1}{\partial x^{r2}} & \frac{\partial x^1}{\partial x^{r3}} \\
		\frac{\partial x^2}{\partial x^{r1}} & \frac{\partial x^2}{\partial x^{r2}} & \frac{\partial x^2}{\partial x^{r3}} \\
		\frac{\partial x^3}{\partial x^{r1}} & \frac{\partial x^3}{\partial x^{r2}} & \frac{\partial x^3}{\partial x^{r3}}
	\end{bmatrix}
\]

Lembrete sobre operadores lineares:

$\ts{F}$ e $\nabla\vt{u}$  são operadores lineares de $E\rightarrow E$. Seja $\ts{T}$ um operador linear:
\[\ts{T}(\alpha\vt{x}+\beta\vt{y})=\alpha\ts{T}\vt{x}+\beta\ts{T}\vt{y},\;\forall\;\vt{x}\text{ e }\vt{y}\in\text{$E$}\]

//Inserir imagem

\[
	\begin{Bmatrix}
		b_1 \\ b_2 \\ b_3
	\end{Bmatrix}
	=
	\underbrace{
	\begin{bmatrix}
		T_{11} & T_{12} & T_{13} \\
		T_{21} & T_{22} & T_{23} \\
		T_{31} & T_{32} & T_{33}
	\end{bmatrix}
	}_{\displaystyle\ts{T}}
	\begin{Bmatrix}
		a_1 \\ a_2 \\ a_3
	\end{Bmatrix}
\]

Obs: Para o curso, um operador linear tem a mesma função de um tensor (simplificando).