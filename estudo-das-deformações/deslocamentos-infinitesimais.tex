Relembrando de $\underline{\mathbf{E}}$ na forma diferencial,
\[\underline{\mathbf{E}}=\frac{1}{2}\left( \frac{\partial u^i}{\partial x^{rj}}+\frac{\partial u^j}{\partial x^{ri}}+\frac{\partial u^k}{\partial x^{ri}}\frac{\partial u^k}{\partial x^{rj}} \right)\]

Para deslocamentos infinitesimais, pode-se desprezar o produto contido na expressão acima. Ficando como,
\[E_{ij}^l=\frac{1}{2}\left( \frac{\partial u^i}{\partial x^{rj}}+\frac{\partial u^j}{\partial x^{ri}}\right)\]

Ou seja, obtém-se o Tensor das Deformações infinitesimais linearizado, denotado por $\underline{\mathbf{E}}^l$ ou $E_{ij}^l$. Sendo,
\[\underline{\mathbf{E}}^l=\frac{1}{2}(\nabla u^{\text{T}}+\nabla u)\]

Agora, lembrando do alongamento linear em função do vetor unitário $\hat{\utilde{\mathbf{m}}}^r$ (direção de uma fibra na configuração de referência),
\[\varepsilon_l(\hat{\utilde{\mathbf{m}}}^r)=\sqrt{1+2(\hat{\utilde{\mathbf{m}}}^r\cdot\underline{\mathbf{E}}\hat{\utilde{\mathbf{m}}}^r)}-1\]

Elevando ambos os lados ao quadrado,
\[(\varepsilon_l+1)^2=1+2(\hat{\utilde{\mathbf{m}}}^r\cdot\underline{\mathbf{E}}\hat{\utilde{\mathbf{m}}}^r)\]
\[\varepsilon_l^2+2\varepsilon_l+1=1+2(\hat{\utilde{\mathbf{m}}}^r\cdot\underline{\mathbf{E}}\hat{\utilde{\mathbf{m}}}^r)\]

Desprezando a ordem superior,
\[\varepsilon_l=\hat{\utilde{\mathbf{m}}}^r\cdot\underline{\mathbf{E}}\hat{\utilde{\mathbf{m}}}^r\]

Ou seja, $\underline{\mathbf{E}}$ tende para $\underline{\mathbf{E}}^l$ para deslocamentos lineares, portanto,
\[\varepsilon_l(\hat{\utilde{\mathbf{m}}}^r)=\hat{\utilde{\mathbf{m}}}^r\cdot\underline{\mathbf{E}}^l\hat{\utilde{\mathbf{m}}}^r\]

E para a distorção,
\[\sin\gamma(\hat{\utilde{\mathbf{a}}}^r, \hat{\utilde{\mathbf{b}}}^r)=\frac{2(\hat{\utilde{\mathbf{b}}}^r\cdot\underline{\mathbf{E}}\hat{\utilde{\mathbf{a}}}^r)}{\sqrt{2(\hat{\utilde{\mathbf{a}}}^r\cdot\underline{\mathbf{E}}\hat{\utilde{\mathbf{a}}}^r)+1}\sqrt{2(\hat{\utilde{\mathbf{b}}}^r\cdot\underline{\mathbf{E}}\hat{\utilde{\mathbf{b}}}^r)+1}}\]

Admitindo que o denominador acima será muito pequeno em relação a $1$,
\[\sin\gamma(\hat{\utilde{\mathbf{a}}}^r, \hat{\utilde{\mathbf{b}}}^r)=2(\hat{\utilde{\mathbf{b}}}^r\cdot\underline{\mathbf{E}}^l\hat{\utilde{\mathbf{a}}}^r)\]

\textbf{Exercício} - Capítulo 3/página 108 do livro \textit{The Mechanics of Solids and Structures - Hierarchical Modeling and The Finite Element Process of Solution (Bucalem, Bathe):}

Configuração deformada para um bloco de aresta $a$,
\[
\begin{cases}
    x^1=x^{r1}+\tan\beta x^{r2} \\ x^2=x^{r2} \\ x^3=x^{r3}
\end{cases}
\]

//Inserir desenho

\begin{enumerate}
    \item Calcular o campo de deslocamentos e deformações normais ($\varepsilon_l$) nas direções ($\utilde{\mathbf{\hat{e}_1}}$), ($\utilde{\mathbf{\hat{e}_2}}$), ($\utilde{\mathbf{m_1}}=\frac{\sqrt{2}}{2}\utilde{\mathbf{\hat{e}_1}}+\frac{\sqrt{2}}{2}\utilde{\mathbf{\hat{e}_2}}$) e ($\utilde{\mathbf{m_2}}=-\frac{\sqrt{2}}{2}\utilde{\mathbf{\hat{e}_1}}+\frac{\sqrt{2}}{2}\utilde{\mathbf{\hat{e}_2}}$).
\item Calcular as deformações por cisalhamento para os pares de fibras nas direções ($\utilde{\mathbf{\hat{e}_1}}$, $\utilde{\mathbf{\hat{e}_2}}$) e ($\utilde{\mathbf{m_1}}$, $\utilde{\mathbf{m_2}}$).
\item Repetir os itens 1 e 2 assumindo um $\beta$ pequeno, obter os resultados usando a teoria dos pequenos deslocamentos, e mostrar que estes mesmos resultados são também obtidos de 1 e 2.
\end{enumerate}

\textbf{Solução}:

\begin{enumerate}
    \item 

    O campo de deslocamentos é dado por,
    \[u^i=x^i-x^{ri}\]

    Aplicando para as coordenadas,
    \[u^1=x^1-x^{r1}=(x^{r1}+\tan\beta x^{r2})-x^{r1}=\tan\beta x^{r2}\]
    \[u^2=x^2-x^{r2}=x^{r2}-x^{r2}=0\]
    \[u^3=x^2-x^{r2}=x^{r3}-x^{r3}=0\]

    Considerando grandes deslocamentos, precisamos calcular o gradiente das deformações ($\underline{\mathbf{F}}$)
    \[
        \underline{\mathbf{F}}
        =
        \begin{bmatrix}
            1 & \tan\beta & 0 \\
            0 & 1 & 0 \\
            0 & 0 & 1
        \end{bmatrix}			
    \]
    Nota-se aqui que através do bloco, os elementos de F são constantes (independentes de $x^{r1}$, $x^{r2}$ e $x^{r3}$).

    Então,
    \[
        \underline{\mathbf{F}}^{\text{T}}\underline{\mathbf{F}}
        =
        \begin{bmatrix}
            1 & 0 & 0 \\
            \tan\beta & 1 & 0 \\
            0 & 0 & 1
        \end{bmatrix}
        \begin{bmatrix}
            1 & \tan\beta & 0 \\
            0 & 1 & 0 \\
            0 & 0 & 1
        \end{bmatrix}
        =
        \begin{bmatrix}
            1 & \tan\beta & 0 \\
            \tan\beta & \tan^2\beta + 1 & 0 \\
            0 & 0 & 1
        \end{bmatrix}
    \]

    Usando a Equação \eqref{eq-alongamento-linear-1} (alongamento linear) para as fibras do item 1:

    Para $\utilde{\mathbf{\hat{e}_1}}$,
    \[\varepsilon_l(\utilde{\mathbf{\hat{e}_1}})=\sqrt{\utilde{\mathbf{\hat{e}_1}}\cdot\underline{\mathbf{C}}\utilde{\mathbf{\hat{e}_1}}}-1\]
    \[
        \utilde{\mathbf{\hat{e}_1}}\cdot\underline{\mathbf{C}}\utilde{\mathbf{\hat{e}_1}}
        =
        \begin{Bmatrix}
            1 & 0 & 0
        \end{Bmatrix}
        \cdot
        \begin{bmatrix}
            1 & \tan\beta & 0 \\
            \tan\beta & \tan^2\beta + 1 & 0 \\
            0 & 0 & 1
        \end{bmatrix}
        \begin{Bmatrix}
            1 \\ 0 \\ 0
        \end{Bmatrix}
    \]
    \[
        \utilde{\mathbf{\hat{e}_1}}\cdot\underline{\mathbf{C}}\utilde{\mathbf{\hat{e}_1}}
        =
        \begin{Bmatrix}
            1 & 0 & 0
        \end{Bmatrix}
        \cdot
        \begin{Bmatrix}
            1 \\ \tan\beta \\ 0
        \end{Bmatrix}
        =
        1
    \]
    \[\varepsilon_l(\utilde{\mathbf{\hat{e}_1}})=\sqrt{1}-1=0\]

    Para $\utilde{\mathbf{\hat{e}_2}}$,
    \[\varepsilon_l(\utilde{\mathbf{\hat{e}_2}})=\sqrt{\utilde{\mathbf{\hat{e}_2}}\cdot\underline{\mathbf{C}}\utilde{\mathbf{\hat{e}_2}}}-1\]
    \[
        \utilde{\mathbf{\hat{e}_2}}\cdot\underline{\mathbf{C}}\utilde{\mathbf{\hat{e}_2}}
        =
        \begin{Bmatrix}
            0 & 1 & 0
        \end{Bmatrix}
        \cdot
        \begin{bmatrix}
            1 & \tan\beta & 0 \\
            \tan\beta & \tan^2\beta + 1 & 0 \\
            0 & 0 & 1
        \end{bmatrix}
        \begin{Bmatrix}
            0 \\ 1 \\ 0
        \end{Bmatrix}
    \]
    \[
        \utilde{\mathbf{\hat{e}_2}}\cdot\underline{\mathbf{C}}\utilde{\mathbf{\hat{e}_2}}
        =
        \begin{Bmatrix}
            0 & 1 & 0
        \end{Bmatrix}
        \cdot
        \begin{Bmatrix}
            \tan\beta \\ \tan^2\beta + 1 \\ 0
        \end{Bmatrix}
        =
        \tan^2\beta + 1
    \]
    \[\varepsilon_l(\utilde{\mathbf{\hat{e}_2}})=\sqrt{\tan^2\beta + 1}-1\]

    Para $\utilde{\mathbf{m_1}}$ (sendo $\utilde{\mathbf{m_1}}$ unitário, obrigatoriamente),
    \[\varepsilon_l(\utilde{\mathbf{m_1}})=\sqrt{\utilde{\mathbf{m_1}}\cdot\underline{\mathbf{C}}\utilde{\mathbf{m_1}}}-1\]
    \[
        \utilde{\mathbf{m_1}}\cdot\underline{\mathbf{C}}\utilde{\mathbf{m_1}}
        =
        \begin{Bmatrix}
            \frac{\sqrt{2}}{2} & \frac{\sqrt{2}}{2} & 0
        \end{Bmatrix}
        \cdot
        \begin{bmatrix}
            1 & \tan\beta & 0 \\
            \tan\beta & \tan^2\beta + 1 & 0 \\
            0 & 0 & 1
        \end{bmatrix}
        \begin{Bmatrix}
            \frac{\sqrt{2}}{2} \\ \frac{\sqrt{2}}{2} \\ 0
        \end{Bmatrix}
    \]
    \[
        \utilde{\mathbf{m_1}}\cdot\underline{\mathbf{C}}\utilde{\mathbf{m_1}}
        =
        \begin{Bmatrix}
            \frac{\sqrt{2}}{2} & \frac{\sqrt{2}}{2} & 0
        \end{Bmatrix}
        \cdot
        \begin{Bmatrix}
            \frac{\sqrt{2}}{2}(\tan\beta+1) \\ \frac{\sqrt{2}}{2}(\tan^2\beta+1+\tan\beta) \\ 0
        \end{Bmatrix}
    \]
    \[\utilde{\mathbf{m_1}}\cdot\underline{\mathbf{C}}\utilde{\mathbf{m_1}}=\frac{1}{2}(\tan\beta+1)+\frac{1}{2}(\tan^2\beta+1+\tan\beta)\]
    \[\utilde{\mathbf{m_1}}\cdot\underline{\mathbf{C}}\utilde{\mathbf{m_1}}=\tan\beta+1+\frac{\tan^2\beta}{2}\]
    \[\varepsilon_l(\utilde{\mathbf{m_1}})=\sqrt{\tan\beta+1+\frac{\tan^2\beta}{2}}-1\]

    Para $\utilde{\mathbf{m_2}}$ (sendo $\utilde{\mathbf{m_2}}$ unitário, obrigatoriamente),
    \[\varepsilon_l(\utilde{\mathbf{m_2}})=\sqrt{\utilde{\mathbf{m_2}}\cdot\underline{\mathbf{C}}\utilde{\mathbf{m_2}}}-1\]
    \[
        \utilde{\mathbf{m_2}}\cdot\underline{\mathbf{C}}\utilde{\mathbf{m_2}}
        =
        \begin{Bmatrix}
            -\frac{\sqrt{2}}{2} & \frac{\sqrt{2}}{2} & 0
        \end{Bmatrix}
        \cdot
        \begin{bmatrix}
            1 & \tan\beta & 0 \\
            \tan\beta & \tan^2\beta + 1 & 0 \\
            0 & 0 & 1
        \end{bmatrix}
        \begin{Bmatrix}
            -\frac{\sqrt{2}}{2} \\ \frac{\sqrt{2}}{2} \\ 0
        \end{Bmatrix}
    \]
    \[
        \utilde{\mathbf{m_2}}\cdot\underline{\mathbf{C}}\utilde{\mathbf{m_2}}
        =
        \begin{Bmatrix}
            -\frac{\sqrt{2}}{2} & \frac{\sqrt{2}}{2} & 0
        \end{Bmatrix}
        \cdot
        \begin{Bmatrix}
            \frac{\sqrt{2}}{2}(\tan\beta-1) \\ \frac{\sqrt{2}}{2}(\tan^2\beta+1-\tan\beta) \\ 0
        \end{Bmatrix}
    \]
    \[\utilde{\mathbf{m_2}}\cdot\underline{\mathbf{C}}\utilde{\mathbf{m_2}}=-\frac{1}{2}(\tan\beta-1)+\frac{1}{2}(\tan^2\beta+1-\tan\beta)\]
    \[\utilde{\mathbf{m_2}}\cdot\underline{\mathbf{C}}\utilde{\mathbf{m_2}}=-\tan\beta+1+\frac{\tan^2\beta}{2}\]
    \[\varepsilon_l(\utilde{\mathbf{m_2}})=\sqrt{1+\frac{\tan^2\beta}{2}-\tan\beta}-1\]

    \item As tensões de cisalhamento podem ser calculadas como segue,
    
    Sabendo que,
    \[\varepsilon_l(\hat{\utilde{\mathbf{m}}}^r)=\sqrt{\hat{\utilde{\mathbf{m}}}^r\cdot\underline{\mathbf{C}}\hat{\utilde{\mathbf{m}}}^r}-1\]
    \[1+\varepsilon_l(\hat{\utilde{\mathbf{m}}}^r)=\sqrt{\hat{\utilde{\mathbf{m}}}^r\cdot\underline{\mathbf{C}}\hat{\utilde{\mathbf{m}}}^r}\]
    
    Substituindo na Equação \eqref{eq-distorcao-fibra-1},
    \[
        \sin\gamma=\frac{\hat{\utilde{\mathbf{b}}}^r\cdot\underline{\mathbf{C}}\hat{\utilde{\mathbf{a}}}^r}{\sqrt{1+\varepsilon_l(\hat{\utilde{\mathbf{a}}}^r)}\sqrt{1+\varepsilon_l(\hat{\utilde{\mathbf{b}}}^r)}}		
    \]
    
    Para o par de fibras ($\utilde{\mathbf{\hat{e}_1}}$, $\utilde{\mathbf{\hat{e}_2}}$),
    \[
        \sin\gamma=\frac{\utilde{\mathbf{\hat{e}_2}}\cdot\underline{\mathbf{C}}\utilde{\mathbf{\hat{e}_1}}}{\sqrt{1+\varepsilon_l(\utilde{\mathbf{\hat{e}_1}})}\sqrt{1+\varepsilon_l(\utilde{\mathbf{\hat{e}_2}})}}
        =
        \frac{\utilde{\mathbf{\hat{e}_2}}\cdot\underline{\mathbf{C}}\utilde{\mathbf{\hat{e}_1}}}{\sqrt{1+\tan^2\beta}}		
    \]
    \[
        \utilde{\mathbf{\hat{e}_2}}\cdot\underline{\mathbf{C}}\utilde{\mathbf{\hat{e}_1}}
        =
        \begin{Bmatrix}
            0 & 1 & 0
        \end{Bmatrix}
        \cdot
        \begin{bmatrix}
            1 & \tan\beta & 0 \\
            \tan\beta & \tan^2\beta + 1 & 0 \\
            0 & 0 & 1
        \end{bmatrix}
        \begin{Bmatrix}
            1 \\ 0 \\ 0
        \end{Bmatrix}
    \]
    \[
        \utilde{\mathbf{\hat{e}_2}}\cdot\underline{\mathbf{C}}\utilde{\mathbf{\hat{e}_1}}
        =
        \begin{Bmatrix}
            0 & 1 & 0
        \end{Bmatrix}
        \cdot
        \begin{Bmatrix}
            1 \\ \tan\beta \\ 0
        \end{Bmatrix}
        =\tan\beta
    \]
    \[
        \sin\gamma=\frac{\tan\beta}{\sqrt{1+\tan^2\beta}}
    \]
    
    Usando a identidade $\sec^2\beta-\tan^2\beta=1$,
    \[
        \sin\gamma=\frac{\tan\beta}{\sqrt{\sec^2\beta}}=\frac{\tan\beta}{\sec\beta}=\frac{\displaystyle\frac{\sin\beta}{\cos\beta}}{\displaystyle\frac{1}{\cos\beta}}=\sin\beta		
    \]
    
    O que implica que $\gamma=\beta$.
    
    
    Para o par de fibras ($\utilde{\mathbf{m_1}}$, $\utilde{\mathbf{m_2}}$),
    \[
        \sin\gamma=\frac{\utilde{\mathbf{m_2}}\cdot\underline{\mathbf{C}}\utilde{\mathbf{m_1}}}{\sqrt{1+\varepsilon_l(\utilde{\mathbf{m_1}})}\sqrt{1+\varepsilon_l(\utilde{\mathbf{m_2}})}}
    \]
    \[
        \utilde{\mathbf{m_2}}\cdot\underline{\mathbf{C}}\utilde{\mathbf{m_1}}
        =
        \begin{Bmatrix}
            -\frac{\sqrt{2}}{2} & \frac{\sqrt{2}}{2} & 0
        \end{Bmatrix}
        \cdot
        \begin{bmatrix}
            1 & \tan\beta & 0 \\
            \tan\beta & \tan^2\beta + 1 & 0 \\
            0 & 0 & 1
        \end{bmatrix}
        \begin{Bmatrix}
            \frac{\sqrt{2}}{2} \\ \frac{\sqrt{2}}{2} \\ 0
        \end{Bmatrix}
    \]
    \[
        \utilde{\mathbf{m_2}}\cdot\underline{\mathbf{C}}\utilde{\mathbf{m_1}}
        =
        \begin{Bmatrix}
            -\frac{\sqrt{2}}{2} & \frac{\sqrt{2}}{2} & 0
        \end{Bmatrix}
        \cdot
        \begin{Bmatrix}
            \frac{\sqrt{2}}{2}(1+\tan\beta) \\ \frac{\sqrt{2}}{2}(\tan\beta+\tan^2\beta+1) \\ 0
        \end{Bmatrix}
    \]
    \[
        \utilde{\mathbf{m_2}}\cdot\underline{\mathbf{C}}\utilde{\mathbf{m_1}}
        =
        -\frac{1}{2}(1+\tan\beta)+\frac{1}{2}(\tan\beta+\tan^2\beta+1)
        =\tan^2\beta
    \]
    \[
        \sin\gamma=\frac{\tan^2\beta}{\sqrt{\tan\beta+1+\displaystyle\frac{\tan^2\beta}{2}}\sqrt{1+\displaystyle\frac{\tan^2\beta}{2}-\tan\beta}}
    \]
    
    \item Quando os deslocamentos são infinitesimais, podemos calcular o alongamento normal e deformação cisalhante usando o tensor infinitesimal das deformações, sendo definido como,
    \[
        E_{ij}
        =\frac{1}{2}\left(\frac{\partial u^i}{\partial x^{rj}}+\frac{\partial u^j}{\partial x^{ri}}\right)
    \]
    
    Então,
    \[E_{11}=\frac{\partial u^1}{\partial x^{r1}}=0,\;E_{22}=\frac{\partial u^2}{\partial x^{r2}}=0,\;E_{33}=\frac{\partial u^3}{\partial x^{r3}}=0\]
    
    E como para deslocamentos infinitesimais $\tan\beta=\beta$,
    \[E_{12}=E_{21}=\frac{1}{2}\left(\frac{\partial u^1}{\partial x^{r2}}+\frac{\partial u^2}{\partial x^{r1}}\right)=\frac{\beta}{2}\]
    \[E_{13}=E_{31}=\frac{1}{2}\left(\frac{\partial u^1}{\partial x^{r3}}+\frac{\partial u^3}{\partial x^{r1}}\right)=0\]
    \[E_{23}=E_{32}=\frac{1}{2}\left(\frac{\partial u^2}{\partial x^{r3}}+\frac{\partial u^3}{\partial x^{r2}}\right)=0\]
    
    Na forma matricial,
    \[
        \underline{\mathbf{E}}
        =
        \begin{bmatrix}
            0 & \beta/2 & 0 \\
            \beta/2 & 0 & 0 \\
            0 & 0 & 0
        \end{bmatrix}	
    \]
    
    Obtendo-se,
    \[
        \varepsilon_l(\utilde{\mathbf{\hat{e}_1}})=E_{11}=0,\;
        \varepsilon_l(\utilde{\mathbf{\hat{e}_2}})=E_{22}=0,\;
        \sin(\utilde{\mathbf{\hat{e}_1}},\;\utilde{\mathbf{\hat{e}_2}})=2E_{12}=\beta
    \]
    
    A deformação normal das fibras nas direções $\utilde{\mathbf{m_1}}$ e $\utilde{\mathbf{m_2}}$ podem ser calculadas por,
    \[\varepsilon_l(\utilde{\mathbf{m_1}})=\utilde{\mathbf{m_1}}\cdot\underline{\mathbf{E}}\utilde{\mathbf{m_1}}\]
    \[
        \varepsilon_l(\utilde{\mathbf{m_1}})
        =
        \begin{Bmatrix}
            \frac{\sqrt{2}}{2} & \frac{\sqrt{2}}{2} & 0
        \end{Bmatrix}
        \cdot
        \begin{bmatrix}
            0 & \beta/2 & 0 \\
            \beta/2 & 0 & 0 \\
            0 & 0 & 0
        \end{bmatrix}
        \begin{Bmatrix}
            \frac{\sqrt{2}}{2} \\ \frac{\sqrt{2}}{2} \\ 0
        \end{Bmatrix}
        =\frac{\beta}{2}
    \]
    \[\varepsilon_l(\utilde{\mathbf{m_2}})=\utilde{\mathbf{m_2}}\cdot\underline{\mathbf{E}}\utilde{\mathbf{m_2}}\]
    \[
        \varepsilon_l(\utilde{\mathbf{m_2}})
        =
        \begin{Bmatrix}
            -\frac{\sqrt{2}}{2} & \frac{\sqrt{2}}{2} & 0
        \end{Bmatrix}
        \cdot
        \begin{bmatrix}
            0 & \beta/2 & 0 \\
            \beta/2 & 0 & 0 \\
            0 & 0 & 0
        \end{bmatrix}
        \begin{Bmatrix}
            -\frac{\sqrt{2}}{2} \\ \frac{\sqrt{2}}{2} \\ 0
        \end{Bmatrix}
        =-\frac{\beta}{2}
    \]
    
    A deformação entre as fibras de direção $\utilde{\mathbf{m_1}}$, $\utilde{\mathbf{m_1}}$ é dada por,
    \[\gamma(\utilde{\mathbf{m_1}},\;\utilde{\mathbf{m_2}})=2(\utilde{\mathbf{m_1}}\cdot\underline{\mathbf{E}}^l\utilde{\mathbf{m_2}})\]
    \[
        \gamma(\utilde{\mathbf{m_1}},\;\utilde{\mathbf{m_2}})
        =
        2
        \begin{Bmatrix}
            \frac{\sqrt{2}}{2} & \frac{\sqrt{2}}{2} & 0
        \end{Bmatrix}
        \cdot
        \begin{bmatrix}
            0 & \beta/2 & 0 \\
            \beta/2 & 0 & 0 \\
            0 & 0 & 0
        \end{bmatrix}
        \begin{Bmatrix}
            -\frac{\sqrt{2}}{2} \\ \frac{\sqrt{2}}{2} \\ 0
        \end{Bmatrix}
        =0
    \]
    
    A fim de mostrar que obtivemos os resultados acima dos valores calculados em 1 e 2, precisamos considerar $\beta$ infinitesimal nas expressões em 1 e 2.
    
    
\end{enumerate}