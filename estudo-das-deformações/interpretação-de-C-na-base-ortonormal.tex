\[\lambda(\utilde{\mathbf{\hat{e}_1}})=\sqrt{\utilde{\mathbf{\hat{e}_1}}\cdot\underline{\mathbf{C}}\utilde{\mathbf{\hat{e}_1}}}=\sqrt{C_{11}}\therefore\lambda^2(\utilde{\mathbf{\hat{e}_1}})=C_{11}\]
\[\lambda(\utilde{\mathbf{\hat{e}_2}})=\sqrt{\utilde{\mathbf{\hat{e}_2}}\cdot\underline{\mathbf{C}}\utilde{\mathbf{\hat{e}_2}}}=\sqrt{C_{22}}\therefore\lambda^2(\utilde{\mathbf{\hat{e}_2}})=C_{22}\]
\[\lambda(\utilde{\mathbf{\hat{e}_3}})=\sqrt{\utilde{\mathbf{\hat{e}_3}}\cdot\underline{\mathbf{C}}\utilde{\mathbf{\hat{e}_3}}}=\sqrt{C_{33}}\therefore\lambda^2(\utilde{\mathbf{\hat{e}_3}})=C_{33}\]

Ou seja, a diagonal de $\underline{\mathbf{C}}$ guarda o alongamento quadrático.

\[\sin(\utilde{\mathbf{\hat{e}_1}},\utilde{\mathbf{\hat{e}_2}})=\frac{\utilde{\mathbf{\hat{e}_1}}\cdot\underline{\mathbf{C}}\utilde{\mathbf{\hat{e}_2}}}{\lambda(\utilde{\mathbf{\hat{e}_1}})\lambda(\utilde{\mathbf{\hat{e}_2}})}=\frac{C_{12}}{\lambda(\utilde{\mathbf{\hat{e}_1}})\lambda(\utilde{\mathbf{\hat{e}_2}})}\]